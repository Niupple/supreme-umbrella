\documentclass[UTF8]{ctexart}
\usepackage{amsmath,amssymb}
\usepackage{fancyhdr}
\usepackage{amsmath,bm}
\usepackage{mathrsfs}
\usepackage{ntheorem}
\usepackage{graphicx}
\usepackage{subfigure}
\usepackage[top=2cm, bottom=2cm, left=2cm, right=2cm]{geometry}  
\usepackage{algorithm}  
\usepackage{algorithmicx}  
\usepackage{algpseudocode}
  
\floatname{algorithm}{算法}  
\renewcommand{\algorithmicrequire}{\textbf{输入:}}  
\renewcommand{\algorithmicensure}{\textbf{输出:}}  
  
\theorembodyfont{\normalfont\rm\CJKfamily{song}}
%\theoremstyle{break}
\newtheorem{theorem}{定理}
\newtheorem{lemma}{引理}
\newtheorem{proposition}{命题}
\newtheorem*{proof}{证}[section]
\newtheorem*{solution}{解}[section]
\title{算法作业3}
\author{丁元杰 17231164}
\date{\today}

\begin{document}
\maketitle

\section{行列均衡问题}
    \subsection*{算法思路}
    尝试构造贪心算法。
    
    记此矩阵$A$的第$i$行$j$列的元素为$a_{ij}$,第$i$行的元素和记为
    $$r_i=\sum_j{a_{ij}}$$
    同时第$j$列的元素和记为
    $$c_j=\sum_i{a_{ij}}$$
    显然地有:
    $$\sum_i{r_i}=\sum_j{c_j}=\sum_{ij}{a_{ij}}$$

    观察可知,限制在此矩阵上的操作只能为元素自增,所以记初始行列和的最大值为$m$,有
    $$m = \max_{i, j}{\{r_i, c_j\}}$$

    现在构造性地给出算法。算法首先按照行连续的方式遍历整个矩阵的所有元素,并对每个元素进行相应的修改。记第$k$次修改后的矩阵为$A^{(k)}$,那么矩阵元素记为$a_{ij}^{(k)}$,对应的行列和为$c_i^{(k)}, r_j^{(k)}$。

    算法进行到$a_{ij}^{(k)}$位置时,考察$c_i^{(k)}$和$r_j^{(k)}$。如果记
    $$\delta^{(k)} := \min{\{m-c_i^{(k)}, m-r_j^{(k)}\}}$$
    则第$k$轮的操作就是对当时的$a_{ij}^{(k)}$加上了$\delta^{(k)}$,即:
    \begin{equation*}
        a_{ij}^{(k+1)}=\begin{cases}
            a_{ij}^{(k)} + \delta^{(k)} & \text{if $k=in+j$} \\
            a_{ij}^{(k)} & \text{else}
        \end{cases}
    \end{equation*}
    可以看出,此算法中的一次操作,相当于题目的$\delta$次操作,所以最后的操作总数即为
    $$ans=\sum_k{\delta^{(k)}}=mn-\sum_{ij}{a_{ij}}$$

    下面证明这个算法的正确性:
    \begin{proof}
        显然地,任何一种满足条件的构造,$mn-\sum_{ij}{a_{ij}}$是它的一个下界。

        首先,在第$k$轮的时候,整个矩阵始终能够保持循环不变性:
        \begin{equation*}
            r_i^{(k)}\leq m, 1\leq i\leq n \\
            c_j^{(k)}\leq m, 1\leq i\leq n
        \end{equation*}

        设在第$k$轮操作$a_{ij}$元素。有:
        $$\delta^{(k)} = \min{\{m-c_i^{(k-1)}, m-r_j^{(k-1)}\}}$$
        $$a_{ij}^{(k)}=a_{ij}^{(k-1)} + \delta^{(k)}$$
        则
        \begin{align*}
            r_i^{(k)} &= r_i^{(k-1)} + \delta^{(k)} \\
            &= r_i^{(k-1)} + \min{\{m-c_i^{(k-1)}, m-r_j^{(k-1)}\}} \\
            &= \min{\{r_i^{(k-1)} + m-c_i^{(k-1)}, r_i^{(k-1)} + m - r_j^{(k-1)}\}} \\
            &= \min{\{r_i^{(k-1)} + m-c_i^{(k-1)}, m\}} \\
            &\leq m
        \end{align*}
        同理可证
        $$c_j^{(k)}\leq m, 1\leq i\leq n$$

        其次,在所有的操作之后,必然有:
        \begin{equation*}
            r_i^{(n^2)} = m, 1\leq i\leq n \\
            c_j^{(n^2)} = m, 1\leq i\leq n
        \end{equation*}

        假设此条件不成立,则必然存在一行$i^*$,其和
        $$r_{i^*}^{(n^2)} < m$$
        那么由恒等式,
        $$\sum_j{c_j}=\sum_i{r_i}<mn$$
        可知必然存在一列$j^*$,其和
        $$c_{j^*}^{(n^2)} < m$$
        然而,在遍历到$(i^*, j^*)$位置时,此二者必有一个依据算法变得不再成立,且由于操作的单调属性,不可能因为其他操作而重新成立。
        所以不可能在整个矩阵中找到相应的一行,即原假设成立。

        综上所述,此算法能够保证在结束之后,有
        \begin{equation*}
            r_i^{(n^2)} = m, 1\leq i\leq n \\
            c_j^{(n^2)} = m, 1\leq i\leq n
        \end{equation*}
        所以此下界为下确界。
    \end{proof}

    \subsection*{伪代码}
    参见算法\ref{algo1}

    \begin{algorithm}
        \caption{求行列均衡问题}
        \begin{algorithmic}[1]
            \Require $A[1..n][1..n]$矩阵
            \Ensure 最小的操作方案
            \Function {CALC}{$A[1..n][1..n]$}
                \State $r[1..n] \gets 0$
                \State $c[1..n] \gets 0$
                \State $m \gets 0$
                \For $i \in [1, n]$
                    \For $j \in [1, n]$
                        \State $r[i] \gets r[i] + A[i][j]$
                        \State $c[j] \gets c[j] + A[i][j]$
                        \State $m \gets$ MAX $(r[i], c[j])$
                    \EndFor
                \EndFor

                \For $i \in [1, n]$
                    \For $j \in [1, n]$
                        \State $\delta \gets$ MIN $(m - r[i], m - c[j])$
                        \State $A[i][j] \gets A[i][j] + \delta$
                        \State $r[i] \gets r[i] + \delta$
                        \State $c[j] \gets c[j] + \delta$
                    \EndFor
                \EndFor

                \State \Return $A$
            \EndFunction
    
        \end{algorithmic}
        \label{algo1}
    \end{algorithm}

    \subsection*{复杂度分析}
    由伪代码可以看出,此算法的复杂度是$O(n^2)$。

\section{数据修改问题} %2
    

\section{最大矩阵问题} %3
    \subsection*{算法思路}
        显然,对于一个答案矩阵,一定满足以下两条性质:
        \begin{enumerate}
            \item 第一列全部是1
            \item 其余每一列的1的个数多于0的个数
        \end{enumerate}

        如若不然,就可以分别进行如下调整:
        \begin{enumerate}
            \item 如果第$i$行第一个元素不为1,那么将这一行取反,答案一定变得更优:
            $$2^m>\sum_{0\leq i < m}{2^i}$$
            \item 如果第$j$列有$a$个1,$b$个0,且有$a < b$,那么将这一列取反,答案将会增加$(b-a)2^{m-j}$。
        \end{enumerate}

        接下来,尝试将初始矩阵朝着目标矩阵的形式调整,分两步进行:
        \begin{enumerate}
            \item 首先将首元素不为1的行全部取反
            \item 对于每一列,如果1的元素多于0的元素,取反
        \end{enumerate}

        可以观察出,这种操作方式得到的最终答案必定是唯一的(虽然矩阵不一定唯一)。原因是,整个操作序列必定是先行后列,在对列进行调整的时候不会再对行进行操作。因为如果在首行全为1的情况下调整某行,又要保证在调整之后使得性质1始终成立,则必须将剩余行全部取反之后对第一列取反。这种操作加上最初的行操作相当于对除了第一列的所有列取反。换而言之,不影响性质1的行操作必然会转为列操作,同时性质1只能通过行操作进行保证,从而说明了朝着目标形式矩阵的调整必定是先行后列的。

        接下来,行调整和列调整的目的都是清楚、且内部互相独立的,因此调整方法是唯一的。

    \subsection*{伪代码}
    参见算法\ref{algo2}。其中,ROW\_REVERSE是对行进行翻转的函数;COLUMN\_REVERSE则是对列进行翻转的函数。BTOI则是将01数组转化为对应的二进制整数的函数。
        \begin{algorithm}
            \caption{求解此问题}
            \begin{algorithmic}[1]
                \Require $A[1..n][1..n]$
                \Ensure 最优方案的值$ans$
                \Function {BestMatrix}{$A[1..n][1..m]$}
                    \For $i \in [1..n]$
                        \If $A[i][1] = 0$
                            \State ROW\_REVERSE($i$)
                        \EndIf
                    \EndFor

                    \For $j \in [1..m]$
                        \State $a \gets 0$
                        \State $b \gets 0$
                        \For $i \in [1..n]$
                            \If $A[i][j] = 1$
                                \State $a \gets a+1$
                            \Else
                                \State $b \gets b+1$
                            \EndIf
                        \EndFor
                        \If $a < b$
                            \State COLUMN\_REVERSE($j$)
                        \EndIf
                    \EndFor

                    \State $ans \gets 0$
                    \For $i \in [1..n]$
                        \State $ans \gets$ BTOI($A[i]$)
                    \EndFor

                    \State \Return $ans$
                \EndFunction
            \end{algorithmic}
            \label{algo2}
        \end{algorithm}

    \subsection*{复杂度分析}
    由伪代码可以看出,此算法的复杂度是$O(nm)$。

\section{环路问题} %4
    \subsection*{算法思路}
        以任意结点为起点进行深度优先搜索(DFS),并且记录当前栈中的结点。

    \subsection*{算法思路}
        有了基本的性质,就可以设计找出最多不用操作的数的算法了。设两个关联数组$l, r:X\to I$其中$X$是输入序列的元素集合,$I$是原序列的指标集。其中$l[x]$表示在所有的$x$中,下标最小的那一个下标;$r[x]$则表示所有序列中的$x$中,下标最大的那一个下标。

        设$A$是最终求得的,不用操作的数的集合,那么可以看出$A$满足几条性质:

        \begin{enumerate}
            \item $\forall x\in X: \min{A} \leq x \leq \max{A} \Rightarrow x\in A$
            \item $\forall x, y \in A: x < y \Rightarrow r[x] < l[y]$
        \end{enumerate}

        其中,第1条性质说明了不操作的数必然是``连续''的一段,这里的连续并不是自然数的连续的含义,而是其中不含有任何其他的$X$的元素的含义,也即在$X$的范围中连续。

        我们再引入一个映射$\sigma: X \to [0..|X|-1]$和它的逆映射$\tau: [0..|X|-1] \to X$,其中$\sigma(x)$表示在$X$中小于$x$的元素数量。有了这个映射,上段所描述的$x, y$在$X$的范围中连续,就可以转化为$\sigma(x), \sigma(y)$在$\mathbb{N}$的范围中连续。

        有了这些记号,我们就能够方便地描述我们的工作。最终所期望的结果,是找出$[0..|X|-1]=:M$的一个最大的连续子集$A^*$,使得下面的性质成立:
        $$\forall x, y \in A^*: r_{\tau(x)} < l_{\tau(y)}$$

        现在使用动态规划的思想解决问题,设$f[i]$表示以$i$结尾的$M$的子区间中,满足上述条件最长子区间长度。由此可以得到一个简略的状态转移函数:

        \begin{equation*}
            f[i] = 
            \begin{cases}
                1 &, i=0 \\
                1 &, r_{\tau({i-1})} > l_{\tau({i})} \\
                f[i-1] + 1 &, r_{\tau({i-1})} < l_{\tau({i})}
            \end{cases}
        \end{equation*}

    \subsection*{伪代码}
        参见算法\ref{algo4}
        \begin{algorithm}
            \caption{求解此问题}
            \begin{algorithmic}[1]
                \Require $A[1..n]$,
                \Ensure 最优方案的值$ans$
                \Function {MinOperation}{$A[1..n]$}
                    \State $f[0..n-1]$
                    \State $l \gets$ Map$\langle$int, int$\rangle$
                    \State $r \gets$ Map$\langle$int, int$\rangle$
                    \For{$i = 1\to n$}
                        \State $l[A[i]]\gets \min{\{l[A[i]], i\}}$
                        \State $r[A[i]]\gets \max{\{l[A[i]], i\}}$
                    \EndFor
                    \State $m \gets l.size()$
                    \State $ll[0..m]$
                    \State $rr[0..m]$
                    \State $i \gets 0$
                    \For{$x\in l$}
                        \State $ll[i]\gets x$
                        \State $i \gets i+1$
                    \EndFor
                    \State $i \gets 0$
                    \For{$x\in r$}
                        \State $rr[i]\gets x$
                        \State $i \gets i+1$
                    \EndFor
                    \State $f[0] \gets 1$
                    \State $ans \gets 0$
                    \For{$i = 1\to m-1$}
                        \If{$rr[i-1] < ll[i]$}
                            \State $f[i] \gets f[i-1]+1$
                        \Else
                            \State $f[i] = 1$
                        \EndIf
                        \State $ans \gets \max{\{ans, f[i]\}}$
                    \EndFor
                    \State \Return $ans$
                \EndFunction
            \end{algorithmic}
            \label{algo4}
        \end{algorithm}    
        
    \subsection*{复杂度分析}
        可以从伪代码中看出,在使用 Map 作为有序关联数组时,可以使用基于平衡二叉树实现的关联数组,从而做到单次$O(\log n)$的访问和修改复杂度。上面有$n$次访问,因此因为关联数组带来的复杂度有$O(n\log n)$。

        综上所述,算法的总体复杂度为$O(n \log n)$

\section{能耗降低问题}
    \subsection*{算法思路和复杂度分析}
        在$k$小于所有字符串中1的个数时,我们无需将任何0变成1,因此期望每一次翻转都对答案有正向的贡献;反之,如果$k$太大,那么答案可以直接记为$k-\#1$。所以我们现在只考虑消除1的情形。
        
        考虑到每一个串在消除掉一定量$k$时,其产生的能耗是一定的。我们可以预处理出每一个串在消除$j$个1之后节省的能耗。如果第$i$个串有$s_i$个1,那么可以把这个串拆成$s_i$件物品,其中第$j$个物品表示第$i$个串去掉$j$个1之后,节省的能耗。这样,以去掉的1的个数为体积,$k$为容量,原问题被转化为了若干个类型的分组背包。

        由于分组背包有着完整的、已解决的实现,我们这里着重关注如何将每一个字符串拆成若干个类型相同的物品。

        对于第$i$个字符串,可以抽取一个单调的指标序列$A_i$,并且有$s_i = |A_i|$。其中,$A_i$储存了$S_i$字符串的所有1出现位置的下标。现在用$w_{ij}$表示第$i$类的第$j$个物品的价值,含义是字符串$S_i$去掉$j$个1之后节省的能耗,其体积为$j$。根据定义
        $$w_{ij} = w_{i0} - \max_{0 \leq k \leq j}{A_i[k+s_i-j]-A_i[k+1]+1}$$
        不难看出,为了计算$w_{ij}$需要花费$O(m^2)$的时间。因此,为了初始化得到所有的物品,需要花费$O(nm^2)$的时间。

        最后,我们得到了$n$组物品,其中每组物品最多$m$件,总容量为$k$,由分组背包的时间复杂度可以知道,用于动态规划的时间是$O(nmk)$。

        综上,总体的时间复杂度
        $$T(n, m, k) = O(nmk+nm^2)$$

    \subsection*{伪代码}
    参见算法\ref{algo5}
    \begin{algorithm}
        \caption{求解此问题}
        \begin{algorithmic}[1]
            \Require $S[1..n][1..m], k$,
            \Ensure 最优方案的值$ans$
            \Function {MinCost}{$S[1..n][1..m], k$}
                \State $w[1..n][1..m]$
                \State $p[1..n]$
                \For{$i=1\to n$}
                    \State $A\gets []$
                    \For{$j=1\to m$}
                        \If{$S[i][j] = \text{`1'}$}
                            \State $A.append(j)$
                        \EndIf
                    \EndFor
                    \State $s\gets A.size()$
                    \State $p[i]\gets s$
                    \State $w[i][0] \gets \max{A}-\min{A}$
                    \For{$j=1\to s$}
                        \State $w[i][j] \gets 0$
                        \For{$l=0\to j$}
                            \State $w[i][j] \gets \max{\{w[i][j], w[i][0]-(A[k+s-j]-A[k+1]+1)\}}$
                        \EndFor
                    \EndFor
                \EndFor
                \State \Return MultiPack($w, p$)
            \EndFunction
        \end{algorithmic}
        \label{algo5}
    \end{algorithm}

\end{document}