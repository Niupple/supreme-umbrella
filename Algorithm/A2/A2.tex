\documentclass[UTF8]{ctexart}
\usepackage{amsmath,amssymb}
\usepackage{fancyhdr}
\usepackage{amsmath,bm}
\usepackage{mathrsfs}
\usepackage{ntheorem}
\usepackage{graphicx}
\usepackage{subfigure}
\usepackage[top=2cm, bottom=2cm, left=2cm, right=2cm]{geometry}  
\usepackage{algorithm}  
\usepackage{algorithmicx}  
\usepackage{algpseudocode}
  
\floatname{algorithm}{算法}  
\renewcommand{\algorithmicrequire}{\textbf{输入:}}  
\renewcommand{\algorithmicensure}{\textbf{输出:}}  
  
\theorembodyfont{\normalfont\rm\CJKfamily{song}}
%\theoremstyle{break}
\newtheorem{theorem}{定理}
\newtheorem{lemma}{引理}
\newtheorem{proposition}{命题}
\newtheorem*{proof}{证}[section]
\newtheorem*{solution}{解}[section]
\title{算法作业}
\author{丁元杰 17231164}
\date{\today}

\begin{document}
\maketitle

\section{最长回文子序列问题}
    \subsection*{算法思路}
    对原序列$S$和原序列的逆序列$S^R$求出最长公共子序列(LCS),可以求得一个长度为$m$的串$T[1..m]$。现在断言$m$即为原问题(LPS)所求序列的长度,且由串$T[1..m/2]+(T[m/2+1..m])^R$拼接而成的新串是原问题的一个解。由此即可将LPS转化为一个等价的LCS问题,由现有的$O(n^2)$的LCS算法求解得到。

    其中,序列$S[1..n]$的逆序列$S^R[1..n]$定义为
    $$S^R[i] = S[n+1-i],\ \forall 1\leq i \leq n$$

    下面给出上面所给断言的形式化描述和严格证明。

    \begin{proposition}
        $S[1..n]$与$S^R[1..n]$可求得一个最长公共子序列,即存在一个指标序列$a[1..m]$,$b[1..m]$,使得$S[a[i]]=S^R[b[i]],\ \forall 1\leq i \leq m$。并且存在另一个$S$的指标序列$c[1..m]$,使得$S[c[i]] = S[c[m+1-i]],\ \forall 1 \leq i \leq m$,并且有
        \begin{equation*}
            c[i] = \begin{cases}
                a[i] &, \text{if $i \leq m/2$} \\
                n+1-b[m+1-i] &, \text{if $i \geq m/2+1$}
            \end{cases}
        \end{equation*}
        并且由此指标集限定的$S$的子序列是它的一个最长回文子序列。
    \end{proposition}

    \begin{proof}
        先证明这个串是一个回文串。可知
        \begin{equation*}
            c[i] = \begin{cases}
                a[i] &, \text{if $i \leq m/2$} \\
                n+1-b[m+1-i] &, \text{if $i \geq m/2+1$}
            \end{cases}
        \end{equation*}
        那么,$\forall 1 \leq i \leq m/2$
        \begin{align*}
            S[c[m+1-i]] 
            &=S[n+1-b[m+1-(m+1-i)]] \\
            &=S[n+1-b[i]]\\
            &=S^R[b[i]]\\
            &=S[a[i]]\\
            &=S[c[i]]
        \end{align*}
        因此,$S[c[i]]$必然是一个回文子序列。

        再证明这个回文串是最长的。如果存在一个更长的子序列,则表示存在一个指标序列$c'[1..m']$,使得$m'>m$且$S[c'[i]] = S[c'[m'+1-i]], \forall 1 \leq i \leq m'$。那么可以构造$a'[i] = c'[i]$,以及$b'[i] = n+1-c'[m'+1-i]$,那么
        \begin{align*}
            S^R[b'[i]]
            &=S^R[n+1-c'[m'+1-i]] \\
            &=S[c'[m'+1-i]] \\
            &=S[c'[i]] \\
            &=S[a[i]]
        \end{align*}
        使得$S[a'[i]]=S^R[b'[i]],\ \forall 1 \leq i \leq m'$。我们成功构造了一个长$m'$的序列,使得其是$S$和$S^R$的公共子序列,此与原假设中关于公共子序列的最长性质矛盾。
    \end{proof}

    由于这是一个构造性的证明,在证明的同时给出了构造的方案,因此可以方便地由求LCS(求解方案)的算法直接得到LPS(求解方案)的算法。

    \subsection*{伪代码}
    参见算法\ref{algo1}

    \begin{algorithm}
        \caption{求序列$S$的最长回文子序列}
        \begin{algorithmic}[1]
            \Require $S[1..n]$序列
            \Ensure 原序列的最长回文子序列$T$
            \Function {LPS}{$S[1..n]$}
                \State $S' \gets$ REVERSE($S$)
                \State $f[1..n][1..n], g[1..n][1..n]$ 是两个数组
                \State $f[0][0] \gets 0, g[0][0] \gets -1$
                \State $ans = 0$
                \State $anp = (-1, -1)$
                \For{$i = 1\to n$}
                    \For{$j = 1\to n$}
                        \If{$S[i] = S'[j]$}
                            \State $f[i][j] \gets f[i-1][j-1] + 1$
                            \State $g[i][j] \gets (i-1, j-1)$
                        \ElsIf{$f[i-1][j] > \max{\{f[i-1][j-1], f[i][j-1]\}}$}
                            \State $f[i][j] \gets f[i-1][j]$
                            \State $g[i][j] \gets (i-1, j)$
                        \ElsIf{$f[i-1][j-1] > \max{\{f[i-1][j], f[i][j-1]\}}$}
                            \State $f[i][j] \gets f[i-1][j-1]$
                            \State $g[i][j] \gets (i-1, j-1)$
                        \ElsIf{$f[i][j-1] > \max{\{f[i-1][j], f[i-1][j-1]\}}$}
                            \State $f[i][j] \gets f[i][j-1]$
                            \State $g[i][j] \gets (i, j-1)$
                        \EndIf

                        \If{$f[i][j] > ans$}
                            \State $ans \gets f[i][j]$
                            \State $anp \gets (i, j)$
                        \EndIf
                    \EndFor
                \EndFor
                \State $T \gets$ GET\_ANSWER($S, g, ansp$)
                \State \Return $T$
            \EndFunction
    
            \Function {GET\_ANSWER}{$S[1..n], g[1..n][1..n], ansp$}
                \State $T \gets$ `'
                \State $nowp \gets ansp$
                \While{$nowp \neq (-1, -1)$}
                    \State $T += S[nowp.first]$
                    \State $nowp \gets g[nowp.first][nowp.second]$
                \EndWhile
                \State \Return REVERSE($T$)
            \EndFunction
        \end{algorithmic}
        \label{algo1}
    \end{algorithm}

\section{餐厅选址问题} %2
    \subsection*{算法思路及复杂度分析}
        \paragraph*{思路}
        考虑动态规划的思想,设$f[i]$表示所选的所有地址中,位置最大为$m_i$的最大收益。设这个局部的最大收益的方案为一个集合$S_i\subseteq [1, n]$,是候选位置下标集的子集。则由假设,
        $$i = \max{S_i}$$
        并且根据题目中对于任意两个餐厅的距离限制,可以得出:
        $$m_i > \max_{i\neq j\in S_i}{m_j} + k$$
        那么我们可以给出转移方程:

        $$f[i] = \max_{j < i \& m_j+k \leq m_i}{f[j]} + p_i$$

        其中如果$\max$所应用的集合为空,那么$\max$得到$0$的结果。

        最终所求的答案,就是

        $$ans=\max_i{f[i]}$$

        这个递推的思路是,如果存在某一个局部的最优解,那么这个局部的最优解必定存在位置最大的地址。去掉这一个位置最大的地址后,剩下的子集必然是满足基本限制的最优解(否则,就可以通过调整获得更优的解法)。

        \paragraph*{朴素实现的复杂度}
        接下来分析一下这个动态规划的朴素实现所需的时间复杂度。可以看出,总计有$f[1..n]$,共$n$个状态。我们的算法顺序求解$1..n$的状态。其中为了求得$f[i]$的答案,需要优先遍历所有${j: j < i}$,检查其是否满足基本限制$m_j+k\leq m_i$,以及最优性。所以$f[i]$所花的时间是$i$。

        那么最后,我们的算法时间复杂度是:
        \begin{align*}
            T(n) 
            &= \sum_{i=1}^{n}{i} \\
            &= \frac{n(n+1)}{2} \\
            &= O(n^2)
        \end{align*}

        所以朴素算法的时间复杂度是$O(n^2)$的。

        \paragraph*{第一次优化}
        $O(n^2)$的复杂度不尽如人意,原因是算法虽然是单调无后效性地计算$f[i]$,较小的$f[j]$却会被反复查询,用于求解最优性。为了解决这个问题,可以引入另一个数组$g[i]$,表示$f[i]$的前缀最大值,即:
        $$g[i] = \max_{j\leq i}{g[i]}$$
        显然,$g[i]$数组随着$i$的增大非严格单调递增,那么只要我们能快速确定$i$状态可以转移的最大前驱$j^*$,就可以直接从$g[j^*]$得到我们需要的前驱状态。即:
        $$j^* = \max_{j\leq i \& m_j+k \leq m_i}{j}$$
        以及
        $$f[i] = g[j^*] + p_i$$
        其中,由于$m_j$的单调性,$j^*$可以通过二分查找,使用$O(\log i)$的时间求得。因此,此优化的最终复杂度变为:
        \begin{align*}
            T(n)
            &= \sum_{i=1}^{n}{\log i} \\
            &= \log{n!} \\
            &\approx n\log n\\
            &= O(n\log n)
        \end{align*}

        \paragraph*{最终优化}
        使用双指针法优化(two pointers)。

        记对状态$i$求出的$j^*$为$j_i$,那么由定义式:
        $$j_i = \max_{j\leq i \& m_j+k \leq m_i}{j}$$
        可以看出$j_i \leq j_{i+1}$,即$j_i$随$i$满足单调性(非严格的)。既然此单调性存在,就可以由$j_i$开始,向后依次枚举,直到$j_{i+l+1}$不再满足性质$m_{j_{i}+l+1} \leq m_i$时,取$j_{i+1} \gets j_{i}+l$,即可得到$j_{i+1}$。

        虽然此方法把二分查找更换为了线性扫描,但是由于$j_{i}$单调递增,总计的尝试次数不会超过$n$,因此无论左指针$j_i$还是右指针$i$,都只会为求解贡献$n$次计算。此方案的时间复杂度为$O(n)$

    \subsection*{伪代码}
    参见算法\ref{algo2}。
    \begin{algorithm}
        \caption{求解此问题}
        \begin{algorithmic}[1]
            \Require $m[1..n], p[1..n]$数组,$k$参数
            \Ensure 最优方案的值$p_m$
            \Function {BestPlaces}{$m[1..n], p[1..n], k$}
                \State $f[1..n]\gets 0$
                \State $g[0..n]\gets 0$
                \State $l \gets 0$
                \State $r \gets 1$
                \While{$r \leq n$}
                    \While{$l+1 < r \&\& m[l+1]+k\leq m[r]$}
                        \State $l\gets l+1$
                    \EndWhile
                    \State $f[r] \gets g[l]+p[r]$
                    \State $g[r] \gets \max{\{f[r], g[r-1]\}}$
                    \State $r\gets r+1$
                \EndWhile
                \State \Return $g[n]$
            \EndFunction
        \end{algorithmic}
        \label{algo2}
    \end{algorithm}

\section{球队组建问题} %3
    \subsection*{算法思路}
        设计动态规划状态$f[i][j]$,其中$1\leq i \leq n$,且$j \in \{0, 1, 2\}$,表示在前$i$列的所有方案中,最后一列的两人选择方案为$j$时的最优解。在这里,$j$的三个取值分别表示其对应二进制数的状态,对应二进制数位为0,表示不选择该人;反之为选择该人。

        转移方向为$f[i][j]$到$f[i+1][j']$,合法的转移参见表
        \begin{table}
            \centering
            \caption{状态转移表格}
            \label{table1}
            \begin{tabular}{|c|c|c|c|}
                & 00 & 01 & 10 \\
                \hline
                00 & 可 & 可 & 可 \\
                \hline
                01 & 可 & 不可 & 可 \\
                \hline
                10 & 可 & 可 & 不可 \\
                \hline
            \end{tabular}
        \end{table}

        在合理的转移方向之间找到最优的转移,答案就在
        $$ans = \max_{j\in\{0, 1, 2\}}{f[n][j]}$$

    \subsection*{伪代码}
        \begin{algorithm}
            \caption{求解此问题}
            \begin{algorithmic}[1]
                \Require $h[2][1..n]$
                \Ensure 最优方案的值$ans$
                \Function {BestTeam}{$h[2][1..n]$}
                    \State $f[0..n][3]$
                    \State $f[0][0] \gets 0$
                    \State $f[0][1] \gets 0$
                    \State $f[0][2] \gets 0$
                    \For{$i=1\to n$}
                        \State $f[i][0] \gets \max{\{f[i-1][0], f[i-1][1], f[i-1][2]\}}$
                        \State $f[i][1] \gets \max{\{f[i-1][0], f[i-1][2]\}} + h[1][i]$
                        \State $f[i][2] \gets \max{\{f[i-1][0], f[i-1][1]\}} + h[2][i]$
                    \EndFor
                    \State \Return $\max{\{f[n][0], f[n][1], f[n][2]\}}$
                \EndFunction
            \end{algorithmic}
            \label{algo2}
        \end{algorithm}

    \subsection*{复杂度分析}
        后面的DP部分,由于循环只对数组进行了$O(n)$次访问,因此总共复杂度为$O(n)$

\section{数组排序问题} %4
    \subsection*{性质观察}
        在正式解决这个问题之前,需要先观察清楚几个问题的固有性质。
        
        \paragraph*{每种数最多只被操作一次}
            如果一个数被操作了两次,那么第一次的操作不会对之后的结果产生任何效果。因为操作的聚集效果和移动效果都具有覆盖的性质,无论第一次操作与否,第二次都可以产生相同的效果。

        \paragraph*{操作用于头插和尾插的数,在时间上具有单调性}
            既然每种数只会被操作一次,操作可以分为头插和尾插,那么可以把所有数分为头插和尾插两类。以头插举例,在所有对数进行头插的操作中,操作的数必然随着时间单调减小,否则不可能造成最后的有序场面;尾插同理。

        \paragraph*{操作用于头插和尾插的数,在值上具有单调性}
            这是在说,如果$x$被用于头插,那么所有满足条件$y<x$的$y$都会被应用头插;如果$x$被用于尾插,那么所有大于$x$的$y$都会被用于尾插。由此,如果将所有的整数预先排序去重,那么它们在整个排序的过程中被应用的操作将呈现
            $$head, head, \dots, non, \dots, tail, \dots$$
            的模样

        \paragraph*{要想获得最少的操作次数,就要找出最多的不用操作的数}
            由上面的几条性质,可以综合得到最后的一条。
    
    \subsection*{算法思路}
        有了基本的性质,就可以设计找出最多不用操作的数的算法了。设两个关联数组$l, r:X\to I$其中$X$是输入序列的元素集合,$I$是原序列的指标集。其中$l[x]$表示在所有的$x$中,下标最小的那一个下标;$r[x]$则表示所有序列中的$x$中,下标最大的那一个下标。

        设$A$是最终求得的,不用操作的数的集合,那么可以看出$A$满足几条性质:

        \begin{enumerate}
            \item $\forall x\in X: \min{A} \leq x \leq \max{A} \Rightarrow x\in A$
            \item $\forall x, y \in A: x < y \Rightarrow r[x] < l[y]$
        \end{enumerate}

        其中,第1条性质说明了不操作的数必然是``连续''的一段,这里的连续并不是自然数的连续的含义,而是其中不含有任何其他的$X$的元素的含义,也即在$X$的范围中连续。

        我们再引入一个映射$\sigma: X \to [0..|X|-1]$和它的逆映射$\tau: [0..|X|-1] \to X$,其中$\sigma(x)$表示在$X$中小于$x$的元素数量。有了这个映射,上段所描述的$x, y$在$X$的范围中连续,就可以转化为$\sigma(x), \sigma(y)$在$\mathbb{N}$的范围中连续。

        有了这些记号,我们就能够方便地描述我们的工作。最终所期望的结果,是找出$[0..|X|-1]=:M$的一个最大的连续子集$A^*$,使得下面的性质成立:
        $$\forall x, y \in A^*: r_{\tau(x)} < l_{\tau(y)}$$

        现在使用动态规划的思想解决问题,设$f[i]$表示以$i$结尾的$M$的子区间中,满足上述条件最长子区间长度。由此可以得到一个简略的状态转移函数:

        \begin{equation*}
            f[i] = 
            \begin{cases}
                1 &, i=0 \\
                1 &, r_{\tau({i-1})} > l_{\tau({i})} \\
                f[i-1] + 1 &, r_{\tau({i-1})} < l_{\tau({i})}
            \end{cases}
        \end{equation*}

    \subsection*{伪代码}
        参见算法\ref{algo4}
        \begin{algorithm}
            \caption{求解此问题}
            \begin{algorithmic}[1]
                \Require $A[1..n]$,
                \Ensure 最优方案的值$ans$
                \Function {MinOperation}{$A[1..n]$}
                    \State $f[0..n-1]$
                    \State $l \gets$ Map$\langle$int, int$\rangle$
                    \State $r \gets$ Map$\langle$int, int$\rangle$
                    \For{$i = 1\to n$}
                        \State $l[A[i]]\gets \min{\{l[A[i]], i\}}$
                        \State $r[A[i]]\gets \max{\{l[A[i]], i\}}$
                    \EndFor
                    \State $m \gets l.size()$
                    \State $ll[0..m]$
                    \State $rr[0..m]$
                    \State $i \gets 0$
                    \For{$x\in l$}
                        \State $ll[i]\gets x$
                        \State $i \gets i+1$
                    \EndFor
                    \State $i \gets 0$
                    \For{$x\in r$}
                        \State $rr[i]\gets x$
                        \State $i \gets i+1$
                    \EndFor
                    \State $f[0] \gets 1$
                    \State $ans \gets 0$
                    \For{$i = 1\to m-1$}
                        \If{$rr[i-1] < ll[i]$}
                            \State $f[i] \gets f[i-1]+1$
                        \Else
                            \State $f[i] = 1$
                        \EndIf
                        \State $ans \gets \max{\{ans, f[i]\}}$
                    \EndFor
                    \State \Return $ans$
                \EndFunction
            \end{algorithmic}
            \label{algo4}
        \end{algorithm}    
        
    \subsection*{复杂度分析}
        可以从伪代码中看出,在使用 Map 作为有序关联数组时,可以使用基于平衡二叉树实现的关联数组,从而做到单次$O(\log n)$的访问和修改复杂度。上面有$n$次访问,因此因为关联数组带来的复杂度有$O(n\log n)$。

        综上所述,算法的总体复杂度为$O(n \log n)$

\section{能耗降低问题}
    \subsection*{算法思路和复杂度分析}
        在$k$小于所有字符串中1的个数时,我们无需将任何0变成1,因此期望每一次翻转都对答案有正向的贡献;反之,如果$k$太大,那么答案可以直接记为$k-\#1$。所以我们现在只考虑消除1的情形。
        
        考虑到每一个串在消除掉一定量$k$时,其产生的能耗是一定的。我们可以预处理出每一个串在消除$j$个1之后节省的能耗。如果第$i$个串有$s_i$个1,那么可以把这个串拆成$s_i$件物品,其中第$j$个物品表示第$i$个串去掉$j$个1之后,节省的能耗。这样,以去掉的1的个数为体积,$k$为容量,原问题被转化为了若干个类型的分组背包。

        由于分组背包有着完整的、已解决的实现,我们这里着重关注如何将每一个字符串拆成若干个类型相同的物品。

        对于第$i$个字符串,可以抽取一个单调的指标序列$A_i$,并且有$s_i = |A_i|$。其中,$A_i$储存了$S_i$字符串的所有1出现位置的下标。现在用$w_{ij}$表示第$i$类的第$j$个物品的价值,含义是字符串$S_i$去掉$j$个1之后节省的能耗,其体积为$j$。根据定义
        $$w_{ij} = w_{i0} - \max_{0 \leq k \leq j}{A_i[k+s_i-j]-A_i[k+1]+1}$$
        不难看出,为了计算$w_{ij}$需要花费$O(m^2)$的时间。因此,为了初始化得到所有的物品,需要花费$O(nm^2)$的时间。

        最后,我们得到了$n$组物品,其中每组物品最多$m$件,总容量为$k$,由分组背包的时间复杂度可以知道,用于动态规划的时间是$O(nmk)$。

        综上,总体的时间复杂度
        $$T(n, m, k) = O(nmk+nm^2)$$

    \subsection*{伪代码}
    参见算法\ref{algo5}
    \begin{algorithm}
        \caption{求解此问题}
        \begin{algorithmic}[1]
            \Require $S[1..n][1..m], k$,
            \Ensure 最优方案的值$ans$
            \Function {MinCost}{$S[1..n][1..m], k$}
                \State $w[1..n][1..m]$
                \State $p[1..n]$
                \For{$i=1\to n$}
                    \State $A\gets []$
                    \For{$j=1\to m$}
                        \If{$S[i][j] = \text{`1'}$}
                            \State $A.append(j)$
                        \EndIf
                    \EndFor
                    \State $s\gets A.size()$
                    \State $p[i]\gets s$
                    \State $w[i][0] \gets \max{A}-\min{A}$
                    \For{$j=1\to s$}
                        \State $w[i][j] \gets 0$
                        \For{$l=0\to j$}
                            \State $w[i][j] \gets \max{\{w[i][j], w[i][0]-(A[k+s-j]-A[k+1]+1)\}}$
                        \EndFor
                    \EndFor
                \EndFor
                \State \Return MultiPack($w, p$)
            \EndFunction
        \end{algorithmic}
        \label{algo5}
    \end{algorithm}

\end{document}