\documentclass[UTF8]{ctexart}
\usepackage{amsmath,amssymb}
\usepackage{fancyhdr}
\usepackage{amsmath,bm}
\usepackage{mathrsfs}
\usepackage{ntheorem}
\usepackage{graphicx}
\usepackage{subfigure}
\usepackage[top=2cm, bottom=2cm, left=2cm, right=2cm]{geometry}  
\usepackage{algorithm}  
\usepackage{algorithmicx}  
\usepackage{algpseudocode}
\usepackage{listings}
\usepackage{palatino}
\usepackage{xcolor}
\usepackage{syntax}
  
\floatname{algorithm}{算法}  
\renewcommand{\algorithmicrequire}{\textbf{输入:}}  
\renewcommand{\algorithmicensure}{\textbf{输出:}}  
  
\lstset{numbers=left, %设置行号位置
        numberstyle=\tiny, %设置行号大小
        keywordstyle=\color{blue}, %设置关键字颜色
        commentstyle=\color[cmyk]{1,0,1,0}, %设置注释颜色
        frame=single, %设置边框格式
        escapeinside=``, %逃逸字符(1左面的键),用于显示中文
        %breaklines, %自动折行
        extendedchars=false, %解决代码跨页时,章节标题,页眉等汉字不显示的问题
        xleftmargin=2em,xrightmargin=2em, aboveskip=1em, %设置边距
        tabsize=4, %设置tab空格数
        showspaces=false %不显示空格
    }
\setlength{\grammarparsep}{10pt plus 1pt minus 1pt} % increase separation between rules
\setlength{\grammarindent}{20em} % increase separation between LHS/RHS 
\theorembodyfont{\normalfont\rm\CJKfamily{song}}
%\theoremstyle{break}
\newtheorem{theorem}{定理}
\newtheorem{lemma}{引理}
\newtheorem{proposition}{命题}
\newtheorem*{proof}{证}[section]
\newtheorem*{solution}{解}[section]
\title{算法作业4}
\author{丁元杰 17231164}
\date{\today}

\begin{document}
\maketitle

\section{判断题}
    \subsection{}
        正确。
    \subsection{}
        错误,现有理论没有证明也没能证否这一点。
    \subsection{}
        正确。
    \subsection{}
        错误,这是必要条件而非充分条件。

\section{最短路问题}
    如图所示的图中,Dijkstra 算法失效,原因如下。最开始出队的两个结点依次为 2 和 1,在这两个点出队以后,其最短路已经确定。但是使用 3 号结点继续更新其邻接的点时,会发现 2 号结点的距离可以进一步地缩短,从而导致 2 号结点需要再次入队。这样就破坏了 Dijkstra 所依赖的假设,从而使 Dijkstra 失效。
    
\section{二分图判定问题}
    \subsection{}
    \begin{proof}
        设二分图 $G=(V, E)$ 中存在一条奇环 $P=(v_0, v_1, \dots, v_{2K+1})$,其中 $v_0=v_{2K+1}$。

        设 $v_0$ 所属的二分图集合为 $V^*$,其中 $V^* = V_1$ 或 $V^* = V_2$,记 $V^*_c = V-V^*$,为 $v_0$ 不在的那个二分图集合。那么由于二分图的限制,显然有 
        \begin{enumerate}
            \item $v_{2k}\in V^*$,其中 $k = 0, 1, \dots, K$
            \item $v_{2k+1}\in V^*_c$,其中 $k = 0, 1, \dots, K$
        \end{enumerate}

        那么得出 $v_0\in V^*$,而 $v_{2K+1}\in V^*_c$,又由于 $v_0 = v_{2K+1}$,所以 $v_0\in V^*_c$。得出 $v_0\in V_1 \cup V_2 \neq \emptyset$。由此得出原图并非二分图。

        所以二分图中不存在奇环。

    \end{proof}

    \subsection{二分图判定算法}

        \subsubsection{基本思想}

        二分图的色数为 2,因此可以使用 0-1 染色的方式判断某一个图是否为一个二分图。这里定义颜色 $c: V\to {-1, 0, 1}$,其中 $-1$ 表示暂时无法确定的颜色,$0, 1$ 则表示二分图中应当填入的颜色。再记录一个集合 $U$,表示已经入队的结点集合,并在入队结束后不进行删除。设某一个结点 $v$,它的颜色为 $c(v)$,与一系列的点相邻 $v_1, v_2, \dots, v_K$,则尝试依次将 $v_k$ 入队,并分情况讨论:

        \begin{enumerate}
            \item $c(v_k) = -1$,则设置它的颜色 $c(v_k) = 1-c(v)$ 后入队;
            \item $c(v_k) /neq -1$ 并且 $c(v_k) = 1-c(v)$,忽略该点;
            \item $c(v_k) /neq -1$ 并且 $c(v_k) = c(v)$,说明发现冲突,该图不是二分图,算法结束。
        \end{enumerate}

        由于该图未必是连通图,在从任意一个结点开始后,需要继续向后扫描未到达的点,以其为入口继续计算。

        如果算法结束还没有发现任何一处冲突,则说明该图为一个二分图。

        \subsubsection{真代码}

        二分图判定方法的真代码如下 \lstinputlisting[language=c++]{algo2.cpp}

        \subsubsection{正确性及复杂度分析}

        如果原图为二分图,BFS 的算法保证了每一个结点至少都会入队一次,从而会出队一次,在每个结点出队之时,都会检查其向外连出的边,对没有分配颜色的点分配颜色,对已经分配颜色的点检查颜色。因此,这种检查之下不会漏掉任何一种二分图中的染色限制,因而如果算法能够得到一个染色,必然是一个合法的染色。由于构造性地给出了一种染色方案,这就说明了原图为二分图。

        此外,如果原图并非二分图,则说明不存在任何一种合法的染色方案,但是显然,对于二分图任何一个连通块,其染色方案只有两种,$c_1, c_2 : V\to {0, 1}$,并且 $c_1 = 1-c_2$。所以任选一个点做起点,任意向其指派一种颜色,便可以推得该连通块上的所有点的颜色,实际上方案只与初始点的颜色相关。而初始点分别为两种颜色的方案具有明显的对称性,同时存在和不存在,因而只需要任取一种颜色作为初始点的颜色,进行如上的检查即可。

        最后关于算法的复杂度,每一个结点都只会被染色一次,也即入队和出队一次。在出队之时,会遍历与之相邻的所有点,因此会遍历图中的每一条边。综上所述,算法的时间复杂度为:

        $$T(V, E) = O(V + E)$$

\section{配对问题}

    \subsection{基本思想}

    首先对所有的对有序化并去重,显然这一步不会影响最后的答案。设操作之后的对偶数量为 $m'$, 集合为 $S$。现在引入一些记号,$d(x)$ 表示包含 $x$ 这个整数的整数对的数量。如果两个整数 $x\neq y$ 是最终的答案,那么必然有 $d(x) + d(y) = m'$,此时 $(x, y)\notin S$;或者有 $d(x) + d(y) = m' + 2$,此时 $(x, y)\in S$。因此,可以考虑预处理 $d$ 的值,并建立按照从 $d$ 到 $x$ 的索引表 $d^{-1}$,然后根据

\section{瓶颈值问题}

    \subsection{算法思路}

    可以借鉴 Prim 的思想,从起点出发构造子图,逐步放宽瓶颈值的条件,直到终点被包含在了子图之中。具体而言,在过程中维护一个大根堆 $H$,其中保存了与当前子图相连的边的编号,按照其边的长短进行排序。记录了一个值 $h$,表示子图中最小的边权为 $h$,并且与子图相邻的所有边都小于 $h$。以此为基础进行拓展。设 $H$ 的堆顶元素(最大元素)为 $x$,其对应的边为 $e_x$,那么:

    \begin{enumerate}
        \item 将 $e_x$ 加入子图中,$h$ 与 $e_x$ 的长度 $d(e_x)$ 取最小值。
        \item 如果与 $e_x$ 相连的另一个节点 $y$ 尚未被加入到子图中,将 $y$ 加入子图,并且将与 $y$ 相邻的所有边入堆。如果此时 $y=T$,是全局的终点,那么算法结束,$h$ 就是最终的瓶颈值。
    \end{enumerate}

    \subsection{真代码}

    该算法的真代码如下 \lstinputlisting[language=c++]{algo5.cpp}

    \subsection{复杂度分析}

    很明显,由于每个结点只会被到达一次,每一个边都会入队一次,因此时间复杂度为:

    $$T(V, E) = O(V + E\log E)$$

\end{document}