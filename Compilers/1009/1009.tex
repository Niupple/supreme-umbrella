\documentclass[UTF8]{ctexart}
\usepackage{amsmath,amssymb}
\usepackage{fancyhdr}
\usepackage{amsmath,bm}
\usepackage{mathrsfs}
\usepackage{ntheorem}
\usepackage{graphicx}
\usepackage{subfigure}
\usepackage[top=2cm, bottom=2cm, left=2cm, right=2cm]{geometry}  
\usepackage{algorithm}  
\usepackage{algorithmicx}  
\usepackage{algpseudocode}
\usepackage{multirow}
\usepackage{tikz}
\usepackage{listings}
\usepackage{xcolor}
\usetikzlibrary{automata, positioning, arrows}
\tikzset{
    ->,
    >=stealth,
    node distance = 3cm,
    every state/.style={thick, fill=gray!10},
    initial text=$ $
}
\lstset{numbers=left, %设置行号位置
        numberstyle=\tiny, %设置行号大小
        keywordstyle=\color{blue}, %设置关键字颜色
        commentstyle=\color[cmyk]{1,0,1,0}, %设置注释颜色
        frame=single, %设置边框格式
        escapeinside=``, %逃逸字符(1左面的键),用于显示中文
        %breaklines, %自动折行
        extendedchars=false, %解决代码跨页时,章节标题,页眉等汉字不显示的问题
        xleftmargin=2em,xrightmargin=2em, aboveskip=1em, %设置边距
        tabsize=4, %设置tab空格数
        showspaces=false %不显示空格
    }
\floatname{algorithm}{算法}  
\renewcommand{\algorithmicrequire}{\textbf{输入:}}  
\renewcommand{\algorithmicensure}{\textbf{输出:}}  

\theorembodyfont{\normalfont\rm\CJKfamily{song}}
%\theoremstyle{break}
\newtheorem{theorem}{定理}
\newtheorem{lemma}{引理}
\newtheorem{proposition}{命题}
\newtheorem*{proof}{证}[section]
\newtheorem*{solution}{解}[section]
\title{符号表作业}
\author{丁元杰 17231164}
\date{\today}

\begin{document}
\maketitle

\section*{5.3}
符号表参见:
\begin{table}[htbp!]
    \centering
    \begin{tabular}{|c|c|c|}
        \hline
        名字&类型&维数\\
        \hline
        ENTRY-OFF&LOGICAL&\\
        ENTRY-ON&LOGICAL&\\
        I&INTEGER&\\
        J&INTEGER&\\
        K&INTEGER&\\
        MIN-VAL-IND&INTEGER&20\\
        LAST1&INTEGER&\\
        LIST-OF-NAMES&STRING&\\
        VAL&REAL&20\\
        X&REAL&\\
        Y&REAL&\\
        Z1&REAL&\\
        Z2&REAL&\\
        Z3&REAL&\\
        \hline
        
    
    \end{tabular}
    
\end{table}

\section*{5.5}
完整表参见\ref{1}:
\begin{table}[htbp!]
    \centering
    \begin{tabular}{|c|c|c|}
        \hline
        编号&名字&属性\\
        \hline
        1&Z&-\\
        2&Y&-\\
        3&J&-\\
        4&S&-\\
        5&FLAG&-\\
        6&Y&-\\
        7&W&-\\
        8&J&-\\
        9&TEST1&-\\
        10&TEST2&-\\
        11&TEST3&-\\
        \hline
        
    \end{tabular}
    \caption{分程序表}
    \label{1}
\end{table}

其中第3段程序即将结束时的栈是\ref{2}:

\begin{table}[htbp!]
    \centering
    \begin{tabular}{|c|}
        \hline
        \\
        6\\
        5\\
        4\\
        3\\
        2\\
        1\\
        \hline
        
    \end{tabular}
    \caption{程序段3结束时的栈}
    \label{2}
\end{table}

第4段程序结束时的栈是\ref{3}:

\begin{table}[htbp!]
    \centering
    \begin{tabular}{|c|}
        \hline
        \\
        11\\
        10\\
        9\\
        8\\
        7\\
        3\\
        2\\
        1\\
        \hline
        
    \end{tabular}
    \caption{程序段4结束时的栈}
    \label{3}
\end{table}

\end{document}