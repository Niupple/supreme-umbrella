\documentclass[UTF8]{ctexart}
\usepackage{amsmath,amssymb}
\usepackage{fancyhdr}
\usepackage{amsmath,bm}
\usepackage{mathrsfs}
\usepackage{ntheorem}
\usepackage{graphicx}
\usepackage{subfigure}
\usepackage[top=2cm, bottom=2cm, left=2cm, right=2cm]{geometry}  
\usepackage{algorithm}  
\usepackage{algorithmicx}  
\usepackage{algpseudocode}
\usepackage{multirow}
\usepackage{tikz}
\usepackage{listings}
\usepackage{xcolor}
\usepackage{syntax}
\usetikzlibrary{automata, positioning, arrows}
\tikzset{
    ->,
    >=stealth,
    node distance = 3cm,
    every state/.style={thick, fill=gray!10},
    initial text=$ $
}
\lstset{numbers=left, %设置行号位置
        numberstyle=\tiny, %设置行号大小
        keywordstyle=\color{blue}, %设置关键字颜色
        commentstyle=\color[cmyk]{1,0,1,0}, %设置注释颜色
        frame=single, %设置边框格式
        escapeinside=``, %逃逸字符(1左面的键),用于显示中文
        %breaklines, %自动折行
        extendedchars=false, %解决代码跨页时,章节标题,页眉等汉字不显示的问题
        xleftmargin=2em,xrightmargin=2em, aboveskip=1em, %设置边距
        tabsize=4, %设置tab空格数
        showspaces=false %不显示空格
    }
\setlength{\grammarparsep}{10pt plus 1pt minus 1pt} % increase separation between rules
\setlength{\grammarindent}{20em} % increase separation between LHS/RHS 
\floatname{algorithm}{算法}  
\renewcommand{\algorithmicrequire}{\textbf{输入:}}  
\renewcommand{\algorithmicensure}{\textbf{输出:}}  

\theorembodyfont{\normalfont\rm\CJKfamily{song}}
%\theoremstyle{break}
\newtheorem{theorem}{定理}
\newtheorem{lemma}{引理}
\newtheorem{proposition}{命题}
\newtheorem*{proof}{证}[section]
\newtheorem*{solution}{解}[section]
\title{Struct属性翻译文法作业}
\author{丁元杰 17231164}
\date{\today}

\begin{document}
\maketitle

\paragraph*{Struct的文法}

\begin{grammar}

    <struct> ::= `struct' `{' <body> `}' `;'

    <body> ::= <declaration> | <declaration> <body>

    <declaration> ::= <type> <identifier list>

    <identifier list> ::= <identifier> `;' | <identifier> `,' <identifier list>

    <identifier> ::= IDENTFIER
    
\end{grammar}

\paragraph*{Struct的属性翻译文法}

\begin{align*}
    \langle \text{struct}\rangle _{\uparrow s_1 \downarrow t_2} &::= \text{`struct'} \text{`\{'} \langle \text{body}\rangle _{\uparrow s_2 \downarrow t_2} \text{`\}'} \text{`;'} \\
    \langle \text{body}\rangle _{\uparrow s_1 \downarrow t_1} &::= \langle \text{declaration}\rangle _{\uparrow new \downarrow old} | \langle \text{declaration}\rangle _{\uparrow new \downarrow old} \langle \text{body}\rangle _{\uparrow s_2 \downarrow t_2} \\
    \langle \text{declaration}\rangle _{\uparrow new \downarrow old} &::= \langle \text{type}\rangle _{\uparrow Type} \langle \text{identifier list}\rangle  _{\uparrow vars} @f_{\uparrow new \downarrow old, vars, Type} \\
    \langle \text{identifier list}\rangle _{\uparrow vars} &::= \langle \text{identifier}\rangle _{\uparrow id} \text{`;'} @item2list_{\uparrow vars \downarrow id} | \langle \text{identifier}\rangle _{\uparrow id} \text{`,'} \langle \text{identifier list}\rangle _{\uparrow vars_old} @addlist_{\uparrow vars \downarrow id} \\
    \langle \text{identifier}\rangle _{\uparrow id} &::= \text{IDENTFIER} @2id_{\uparrow id}
\end{align*}

\end{document}