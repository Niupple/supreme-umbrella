\documentclass[UTF8]{ctexart}
\usepackage{amsmath,amssymb}
\usepackage{fancyhdr}
\usepackage{amsmath,bm}
\usepackage{mathrsfs}
\usepackage{ntheorem}
\usepackage{graphicx}
\usepackage{subfigure}
\usepackage[top=2cm, bottom=2cm, left=2cm, right=2cm]{geometry}  
\usepackage{algorithm}  
\usepackage{algorithmicx}  
\usepackage{algpseudocode}
\usepackage{multirow}
\usepackage{tikz}
\usepackage{listings}
\usepackage{palatino}
\usepackage{xcolor}
\usepackage{syntax}
\usetikzlibrary{automata, positioning, arrows, shapes.geometric}
\tikzstyle{startstop} = [rectangle, minimum width = 2cm, minimum height=1cm,text centered, draw = black]
\tikzstyle{io} = [trapezium, trapezium left angle=70, trapezium right angle=110, minimum width=2cm, minimum height=1cm, text centered, draw=black]
\tikzstyle{process} = [rectangle, minimum width=3cm, minimum height=1cm, text centered, draw=black]
\tikzstyle{arrow} = [->,>=stealth]
\tikzstyle{decision} = [diamond, aspect = 3, text centered, draw=black]
\tikzset{
    ->,
    >=stealth,
    node distance = 3cm,
    every state/.style={thick, fill=gray!10},
    initial text=$ $
}
\lstset{numbers=left, %设置行号位置
        numberstyle=\tiny, %设置行号大小
        keywordstyle=\color{blue}, %设置关键字颜色
        commentstyle=\color[cmyk]{1,0,1,0}, %设置注释颜色
        frame=single, %设置边框格式
        escapeinside=``, %逃逸字符(1左面的键),用于显示中文
        %breaklines, %自动折行
        extendedchars=false, %解决代码跨页时,章节标题,页眉等汉字不显示的问题
        xleftmargin=2em,xrightmargin=2em, aboveskip=1em, %设置边距
        tabsize=4, %设置tab空格数
        showspaces=false %不显示空格
    }
\setlength{\grammarparsep}{10pt plus 1pt minus 1pt} % increase separation between rules
\setlength{\grammarindent}{20em} % increase separation between LHS/RHS 
\floatname{algorithm}{算法}  
\renewcommand{\algorithmicrequire}{\textbf{输入:}}  
\renewcommand{\algorithmicensure}{\textbf{输出:}}  

\theorembodyfont{\normalfont\rm\CJKfamily{song}}
%\theoremstyle{break}
\newtheorem{theorem}{定理}
\newtheorem{lemma}{引理}
\newtheorem{proposition}{命题}
\newtheorem*{proof}{证}[section]
\newtheorem*{solution}{解}[section]
\title{代码优化作业3}
\author{丁元杰 17231164}
\date{\today}

\begin{document}
\maketitle

% 箭头形式

\section*{14.6}

到达定义流:

\begin{align*}
    gen[B1] &= \{<B1, 1>, <B1, 2>, <B1, 3>\} \\
    kill[B1] &= \{<B1, 1>, <B1, 2>, <B3, 1>\} \\
    gen[B2] &= \{<B2, 1>\} \\
    kill[B2] &= \{<B1, 1>, <B1, 2>, <B3, 1>\} \\
    gen[B3] &= \{<B3, 1>, <B3, 2>\} \\
    kill[B3] &= \{<B1, 1>, <B1, 2>, <B3, 1>\} \\
    gen[B4] &= \{<B4, 1>\} \\
    kill[B4] &= \{<B1, 1>, <B1, 2>, <B3, 1>\}
\end{align*}

迭代后结果:

\begin{align*}
    in[B1] &= \{\} \\
    out[B1] &= \{<B1, 1>, <B1, 2>, <B1, 3>\} \\
    in[B2] &= \{<B1, 1>, <B1, 2>, <B1, 3>\} \\
    out[B2] &= \{<B2, 1>, <B1, 3>\} \\
    in[B3] &= \{<B1, 1>, <B1, 2>, <B1, 3>\} \\
    out[B3] &= \{<B3, 1>, <B3, 2>, <B1, 3>\} \\
    in[B4] &= \{<B2, 1>, <B3, 1>, <B3, 2>, <B1, 3>\} \\
    out[B4] &= \{<B1, 3>, <B2, 1>, <B3, 2>\}
\end{align*}

使用-定义链数据流:

\begin{align*}
    a: &L1 = \{<B2, 1>\} \\
    b: &L2 = \{<B1, 1>, <B1, 3>, <B2, 1>, <B4, 1>\} \\
       &L3 = \{<B3, 1>, <B3, 2>, <B4, 1>\} \\
    c: &L4 = \{<B1, 2>, <B2, 1>, <B4, 1>\} \\
       &L5 = \{<B1, 2>, <B3, 2>, <B4, 1>\} \\
    d: &L6 = \{<B1, 3>\} \\
       &L7 = \{<B3, 1>\} \\
    e: &L8 = \{<B4, 1>\} \\
\end{align*}

\end{document}