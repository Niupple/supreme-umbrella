\documentclass[UTF8]{ctexart}
\usepackage{amsmath,amssymb}
\usepackage{fancyhdr}
\usepackage{amsmath,bm}
\usepackage{mathrsfs}
\usepackage{ntheorem}
\usepackage{graphicx}
\usepackage{subfigure}
\usepackage[top=2cm, bottom=2cm, left=2cm, right=2cm]{geometry}  
\usepackage{algorithm}  
\usepackage{algorithmicx}  
\usepackage{algpseudocode}
\usepackage{multirow}
\usepackage{tikz}
\usepackage{listings}
\usepackage{xcolor}
\usetikzlibrary{trees}
\usetikzlibrary{automata, positioning, arrows}
\tikzset{
    ->,
    >=stealth,
    node distance = 3cm,
    every state/.style={thick, fill=gray!10},
    initial text=$ $
}
\lstset{numbers=left, %设置行号位置
        numberstyle=\tiny, %设置行号大小
        keywordstyle=\color{blue}, %设置关键字颜色
        commentstyle=\color[cmyk]{1,0,1,0}, %设置注释颜色
        frame=single, %设置边框格式
        escapeinside=``, %逃逸字符(1左面的键),用于显示中文
        %breaklines, %自动折行
        extendedchars=false, %解决代码跨页时,章节标题,页眉等汉字不显示的问题
        xleftmargin=2em,xrightmargin=2em, aboveskip=1em, %设置边距
        tabsize=4, %设置tab空格数
        showspaces=false %不显示空格
    }
\floatname{algorithm}{算法}  
\renewcommand{\algorithmicrequire}{\textbf{输入:}}  
\renewcommand{\algorithmicensure}{\textbf{输出:}}  

\theorembodyfont{\normalfont\rm\CJKfamily{song}}
%\theoremstyle{break}
\newtheorem{theorem}{定理}
\newtheorem{lemma}{引理}
\newtheorem{proposition}{命题}
\newtheorem*{proof}{证}[section]
\newtheorem*{solution}{解}[section]
\title{中间表示法 作业}
\author{丁元杰 17231164}
\date{\today}

\begin{document}
\maketitle

\section*{7.1}
表示
X Y 2 + = label1 BZ Z X :=  label2 BR Z Y 1 + :=

\section*{7.2}
\subsection*{三元表达式}
\begin{align*}
    \textcircled{1}: & +, B, C \\
    \textcircled{2}: & \uparrow, \textcircled{1}, E\\
    \textcircled{3}: & +, B, C \\
    \textcircled{4}: & *, \textcircled{3}, F\\
    \textcircled{5}: & +, \textcircled{3}, \textcircled{4}\\
    \textcircled{6}: & :=, A, \textcircled{5}
\end{align*}

\subsection*{间接三元式}

三元式:
\begin{align*}
    \textcircled{1}: & +, B, C \\
    \textcircled{2}: & \uparrow, \textcircled{1}, E\\
    \textcircled{3}: & *, \textcircled{1}, F\\
    \textcircled{4}: & +, \textcircled{1}, \textcircled{3}\\
    \textcircled{5}: & :=, A, \textcircled{4}
\end{align*}

操作:
\begin{align*}
    1. \textcircled{1} \\
    2. \textcircled{2} \\
    3. \textcircled{1} \\
    4. \textcircled{3} \\
    5. \textcircled{4} \\
    6. \textcircled{5} \\
\end{align*}

\subsection*{四元式}

\begin{align*}
    +, B, C, T_1 \\
    \uparrow, T_1, E, T_2 \\
    *, T_1, F, T_3 \\
    +, T_3, T_2, A 
\end{align*}

\section*{7.3}

四元式:

①
\begin{align*}
    +, A, 1, T_1 \\
    *=, B, 0, T_1
\end{align*}

②
\begin{align*}
    [], A, 1, T_1 \\
    :=, T_1, 0, B
\end{align*}

\section*{7.4}
\begin{tikzpicture}
	[thick,scale=1, every node/.style={scale=2}]
	\node {root}
	child {node {1}
		child {node {2}
			child {node {3}}
		}
		child [missing] {}
		child {node {3}
			child {node {2}}
		}
	}	
	child [missing] {}	
	child [missing] {}
	child [missing] {}	
	child [missing] {}	
	child { node {2}
		child {node {1}
			child {node {3}}
		}
		child [missing] {}
		child {node {3}
			child {node {1}}
		}
	}	
	child [missing] {}	
	child [missing] {}
	child [missing] {}	
	child [missing] {}	
	child { node {3}
		child {node {1}
			child {node {2}}
		}
		child [missing] {}
		child {node {2}
			child {node {1}}
		}
	};
\end{tikzpicture}

\end{document}