\documentclass[UTF8]{ctexart}
\usepackage{amsmath,amssymb}
\usepackage{fancyhdr}
\usepackage{amsmath,bm}
\usepackage{mathrsfs}
\usepackage{ntheorem}
\usepackage{graphicx}
\usepackage{subfigure}
\usepackage[top=2cm, bottom=2cm, left=2cm, right=2cm]{geometry}  
\usepackage{algorithm}  
\usepackage{algorithmicx}  
\usepackage{algpseudocode}
\usepackage{multirow}
\usepackage{tikz}
\usepackage{listings}
\usepackage{xcolor}
\usetikzlibrary{automata, positioning, arrows}
\tikzset{
    ->,
    >=stealth,
    node distance = 3cm,
    every state/.style={thick, fill=gray!10},
    initial text=$ $
}
\lstset{numbers=left, %设置行号位置
        numberstyle=\tiny, %设置行号大小
        keywordstyle=\color{blue}, %设置关键字颜色
        commentstyle=\color[cmyk]{1,0,1,0}, %设置注释颜色
        frame=single, %设置边框格式
        escapeinside=``, %逃逸字符(1左面的键),用于显示中文
        %breaklines, %自动折行
        extendedchars=false, %解决代码跨页时,章节标题,页眉等汉字不显示的问题
        xleftmargin=2em,xrightmargin=2em, aboveskip=1em, %设置边距
        tabsize=4, %设置tab空格数
        showspaces=false %不显示空格
    }
\floatname{algorithm}{算法}  
\renewcommand{\algorithmicrequire}{\textbf{输入:}}  
\renewcommand{\algorithmicensure}{\textbf{输出:}}  

\theorembodyfont{\normalfont\rm\CJKfamily{song}}
%\theoremstyle{break}
\newtheorem{theorem}{定理}
\newtheorem{lemma}{引理}
\newtheorem{proposition}{命题}
\newtheorem*{proof}{证}[section]
\newtheorem*{solution}{解}[section]
\title{语法制导的翻译作业}
\author{丁元杰 17231164}
\date{\today}

\begin{document}
\maketitle

\section*{9-1.1}
文法如下:
\begin{align*}
    E &\to @ +E+T\\
    E &\to T \\
    T &\to @ *T*F\\
    T &\to F
\end{align*}

\section*{9-2.2}
\subsection*{(1)}
产生输入符号串的倒置:
\begin{align*}
    S &\to \varepsilon\\
    S &\to 0S\ @0\\
    S &\to 1S\ @1
\end{align*}

\subsection*{(2)}
产生空串:
\begin{align*}
    S &\to \varepsilon\ @\varepsilon\\
    S &\to 0S\\
    S &\to 1S
\end{align*}

\subsection*{(3)}
产生符号串本身:
\begin{align*}
    S &\to \varepsilon\\
    S &\to @0\ 0S\\
    S &\to @1\ 1S
\end{align*}

\subsection*{(4)}
产生排序后串:
\begin{align*}
    S &\to \varepsilon\\
    S &\to @0\ 0S\\
    S &\to 1S\ @1
\end{align*}

\section*{9-1.3}
作用是匹配字符串 ``ENGLISH'',并且输出字符串 ``CHINESE''

\section*{9-1.4}
这是一个简单赋值形式的L-属性翻译文法。

\section*{9-1.5}
所有对偶如下:
\begin{itemize}
    \item $\langle dcb, @y@x@b\rangle$
    \item $\langle dxcb, @x@y\rangle$
    \item $\langle baxcb, @x@y@x\rangle$
\end{itemize}

\section*{9-3.2}
是L-属性文法。

\begin{enumerate}
    \item $r$受$q, u, t$影响,$p$受$q, u$影响。
    \item $r$受$q, u$影响,$p$受$q$影响。
    \item $r$受$q$影响,$p$受$q, u$影响。
\end{enumerate}

\section*{9-3.3}
重写如下:
\begin{enumerate}
    \item $\langle S\rangle_{\uparrow S1\downarrow S2}\to\langle A\rangle_{\uparrow S2}@f_{1\uparrow I2\downarrow I1, S2}\langle B\rangle_{\uparrow S3\downarrow I1}@f_{2\uparrow I3\downarrow I1} \langle C\rangle_{\uparrow S4\downarrow I3}@f_{3\uparrow I4\downarrow I2, I3}\langle D \rangle_{\uparrow S5 \downarrow I4}@g_{\uparrow S1 \downarrow S2}$ \\
    $I2 := f_1(I1, S2),\ I3 := f_2(I1),\ I4 := f_3(I2, I3),\ S1 := g(S2)$
    \item $\langle S \rangle_{\uparrow S1 \downarrow I1} \to \varepsilon @f_{\uparrow S1 \downarrow I1}$ \\
    $S1 := f(I1)$
    \item $\langle S \rangle_{\uparrow S1, S2} \to \langle A \rangle_{\uparrow S3}@f_{\uparrow I1, I2, I3, I4 \downarrow S3} @JOHN_{\downarrow I2, I3} \langle C \rangle_{\downarrow I_4} @g_{\uparrow S1 \downarrow I4}$ \\
    $I1, I2, I3, I4 := f(S3),\ S1 := g(I_4)$
\end{enumerate}

\end{document}