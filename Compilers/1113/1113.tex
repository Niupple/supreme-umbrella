\documentclass[UTF8]{ctexart}
\usepackage{amsmath,amssymb}
\usepackage{fancyhdr}
\usepackage{amsmath,bm}
\usepackage{mathrsfs}
\usepackage{ntheorem}
\usepackage{graphicx}
\usepackage{subfigure}
\usepackage[top=2cm, bottom=2cm, left=2cm, right=2cm]{geometry}  
\usepackage{algorithm}  
\usepackage{algorithmicx}  
\usepackage{algpseudocode}
\usepackage{multirow}
\usepackage{tikz}
\usepackage{listings}
\usepackage{palatino}
\usepackage{xcolor}
\usepackage{syntax}
\usetikzlibrary{automata, positioning, arrows, shapes.geometric}
\tikzstyle{startstop} = [rectangle, minimum width = 2cm, minimum height=1cm,text centered, draw = black]
\tikzstyle{io} = [trapezium, trapezium left angle=70, trapezium right angle=110, minimum width=2cm, minimum height=1cm, text centered, draw=black]
\tikzstyle{process} = [rectangle, minimum width=3cm, minimum height=1cm, text centered, draw=black]
\tikzstyle{arrow} = [->,>=stealth]
\tikzstyle{decision} = [diamond, aspect = 3, text centered, draw=black]
\tikzset{
    ->,
    >=stealth,
    node distance = 3cm,
    every state/.style={thick, fill=gray!10},
    initial text=$ $
}
\lstset{numbers=left, %设置行号位置
        numberstyle=\tiny, %设置行号大小
        keywordstyle=\color{blue}, %设置关键字颜色
        commentstyle=\color[cmyk]{1,0,1,0}, %设置注释颜色
        frame=single, %设置边框格式
        escapeinside=``, %逃逸字符(1左面的键),用于显示中文
        %breaklines, %自动折行
        extendedchars=false, %解决代码跨页时,章节标题,页眉等汉字不显示的问题
        xleftmargin=2em,xrightmargin=2em, aboveskip=1em, %设置边距
        tabsize=4, %设置tab空格数
        showspaces=false %不显示空格
    }
\setlength{\grammarparsep}{10pt plus 1pt minus 1pt} % increase separation between rules
\setlength{\grammarindent}{20em} % increase separation between LHS/RHS 
\floatname{algorithm}{算法}  
\renewcommand{\algorithmicrequire}{\textbf{输入:}}  
\renewcommand{\algorithmicensure}{\textbf{输出:}}  

\theorembodyfont{\normalfont\rm\CJKfamily{song}}
%\theoremstyle{break}
\newtheorem{theorem}{定理}
\newtheorem{lemma}{引理}
\newtheorem{proposition}{命题}
\newtheorem*{proof}{证}[section]
\newtheorem*{solution}{解}[section]
\title{代码生成作业}
\author{丁元杰 17231164}
\date{\today}

\begin{document}
\maketitle

% 箭头形式

\section*{15.1}

    \subsection*{(1)}
        栈式:
        \begin{lstlisting}
PUSH B
PUSH C
ADD
PUSH A
MUL
POP T1
PUSH B
PUSH C
ADD
POP T2
PUSH D
PUSH T2
DIV
POP T2
PUSH T1
PUSH T2
SUB
POP X
        \end{lstlisting}

        累加式:
        \begin{lstlisting}
LOAD B
ADD C
STORE T1
LOAD A
MUL T1
STORE T1
LOAD B
ADD C
STORE T2
LOAD D
DIV T2
STORE T2
LOAD T1
SUB T2
LOAD X
        \end{lstlisting}
    
        寄存器-内存式:
        \begin{lstlisting}
LOAD R0, B
ADD R1, R0, C
MUL R1, A, R1
LOAD R2, B
ADD R2, R2, C
DIV R3, D, R2
SUB R0, R1, R3
STORE R0, X
        \end{lstlisting}

        寄存器-寄存器式:
        \begin{lstlisting}
LOAD R0, B
LOAD R1, C
ADD R0, R0, R1
LOAD R1, A
MULT R0, R0, R1
LOAD R1, B
LOAD R2, C
ADD R1, R1, R2
LOAD R2, D
DIV R2, R2, R1
SUB R0, R0, R2
STORE R0, X
        \end{lstlisting}

    \subsection*{(2)}
        栈式:$3\times 18=54$周期

        累加式:$3\times 15=45$周期
        
        寄存器-内存式:$3\times 7 + 1\times 1=22$周期
        
        寄存器-寄存器式:$3\times 7 + 1\times 5=26$周期

\section*{15.4}
    如果分配在静态区,那么执行结果将会是:
    \begin{lstlisting}
0
1
    \end{lstlisting}

    而如果分配在栈上,执行结果变成:
    \begin{lstlisting}
0
0
    \end{lstlisting}

    原因是C语言中,栈上分配的局部变量会在每次调用时分配内存,同时初始化,因此每次都会被清零;而分配在静态区的变量,其生命周期与程序同长,因而只有一次初始化,第二次使用时则直接引用静态区存储的值。

\section*{15.5}

    分配过程:
    \begin{enumerate}
        \item 选择$b$,因为它的度数小于$2$
        \item 选择$x$,指定不为其分配寄存器
        \item 选择$a$,指定不为其分配寄存器
        \item 选择$y$,将其移走,因为度数小于$2$。现在只剩一个结点$i$
        \item 将$i$分配在寄存器$R0$上
        \item 将$y$加入图中,分配在寄存器$R1$上
        \item 将$b$加入图中,分配在寄存器$R0$上
    \end{enumerate}

    最后的分配结果:
    \begin{itemize}
        \item $i, b$,使用$R0$寄存器
        \item $y$,使用$R1$寄存器
        \item $a, x$,不使用全局寄存器
    \end{itemize}

\section*{15.6}
    四元式
    \begin{lstlisting}
T1 = B + C
T2 = A * T1
T3 = B + C
T4 = D * T3
T5 = T2 - T4
    \end{lstlisting}

    汇编代码:
    \begin{lstlisting}
mov     edx, DWORD PTR [esp + 10]
mov     eax, DWORD PTR [esp + 14]
add     eax, edx
imul    eax, DWORD PTR [esp + 18]
mov     edx, eax
mov     ecx, DWORD PTR [esp + 10]
mov     eax, DWORD PTR [esp + 14]
add     eax, ecx
imul    eax, DWORD PTR [esp + 1C]
sub     edx, eax
mov     eax, edx
    \end{lstlisting}

\end{document}