\documentclass[UTF8]{ctexart}
\usepackage{amsmath,amssymb}
\usepackage{fancyhdr}
\usepackage{amsmath,bm}
\usepackage{mathrsfs}
\usepackage{ntheorem}
\usepackage{graphicx}
\usepackage{subfigure}
\usepackage[top=2cm, bottom=2cm, left=2cm, right=2cm]{geometry}  
\usepackage{algorithm}  
\usepackage{algorithmicx}  
\usepackage{algpseudocode}
\usepackage{multirow}
\usepackage{tikz}
\usepackage{listings}
\usepackage{palatino}
\usepackage{xcolor}
\usepackage{syntax}
\usetikzlibrary{automata, positioning, arrows, shapes.geometric}
\tikzstyle{startstop} = [rectangle, minimum width = 2cm, minimum height=1cm,text centered, draw = black]
\tikzstyle{io} = [trapezium, trapezium left angle=70, trapezium right angle=110, minimum width=2cm, minimum height=1cm, text centered, draw=black]
\tikzstyle{process} = [rectangle, minimum width=3cm, minimum height=1cm, text centered, draw=black]
\tikzstyle{arrow} = [->,>=stealth]
\tikzstyle{decision} = [diamond, aspect = 3, text centered, draw=black]
\tikzset{
    ->,
    >=stealth,
    node distance = 3cm,
    every state/.style={thick, fill=gray!10},
    initial text=$ $
}
\lstset{numbers=left, %设置行号位置
        numberstyle=\tiny, %设置行号大小
        keywordstyle=\color{blue}, %设置关键字颜色
        commentstyle=\color[cmyk]{1,0,1,0}, %设置注释颜色
        frame=single, %设置边框格式
        escapeinside=``, %逃逸字符(1左面的键),用于显示中文
        %breaklines, %自动折行
        extendedchars=false, %解决代码跨页时,章节标题,页眉等汉字不显示的问题
        xleftmargin=2em,xrightmargin=2em, aboveskip=1em, %设置边距
        tabsize=4, %设置tab空格数
        showspaces=false %不显示空格
    }
\setlength{\grammarparsep}{10pt plus 1pt minus 1pt} % increase separation between rules
\setlength{\grammarindent}{20em} % increase separation between LHS/RHS 
\floatname{algorithm}{算法}  
\renewcommand{\algorithmicrequire}{\textbf{输入:}}  
\renewcommand{\algorithmicensure}{\textbf{输出:}}  

\theorembodyfont{\normalfont\rm\CJKfamily{song}}
%\theoremstyle{break}
\newtheorem{theorem}{定理}
\newtheorem{lemma}{引理}
\newtheorem{proposition}{命题}
\newtheorem*{proof}{证}[section]
\newtheorem*{solution}{解}[section]
\title{高级语法分析}
\author{丁元杰 17231164}
\date{\today}

\begin{document}
\maketitle

% 箭头形式

\section*{12-3}

    \subsection*{(1)}
        $E+i*i\#$ 的活前缀集合为:
        $$\{E, E+, E+i\}$$

        $E+P\uparrow(i+i)$ 的活前缀集合为:
        $$\{E, E+, E+P, E+P\uparrow, E+P\uparrow(, E+P\uparrow(i\}$$

    \subsection*{(2)}

        $i+i*i$: 
        \begin{align*}
            S 
            & \Rightarrow E\# \\
            & \Rightarrow E+T\# \\
            & \Rightarrow E+P\# \\
            & \Rightarrow E+P*F\# \\
            & \Rightarrow E+P*i\# \\
            & \Rightarrow E+F*i\# \\
            & \Rightarrow E+i*i\# \\
            & \Rightarrow T+i*i\# \\
            & \Rightarrow P+i*i\# \\
            & \Rightarrow F+i*i\# \\
            & \Rightarrow i+i*i\#
        \end{align*}

        $i+i\uparrow(i+i)$
        \begin{align*}
            S
            & \Rightarrow E\# \\
            & \Rightarrow E+T\# \\
            & \Rightarrow E+P\uparrow T \\
            & \Rightarrow E+P\uparrow P \\
            & \Rightarrow E+P\uparrow F \\
            & \Rightarrow E+P\uparrow (E) \\
            & \Rightarrow E+P\uparrow (E+T) \\
            & \Rightarrow E+P\uparrow (E+P) \\
            & \Rightarrow E+P\uparrow (E+F) \\
            & \Rightarrow E+P\uparrow (E+i) \\
            & \Rightarrow E+P\uparrow (T+i) \\
            & \Rightarrow E+P\uparrow (P+i) \\
            & \Rightarrow E+P\uparrow (F+i) \\
            & \Rightarrow E+P\uparrow (i+i) \\
            & \Rightarrow E+F\uparrow (i+i) \\
            & \Rightarrow E+i\uparrow (i+i) \\
            & \Rightarrow T+i\uparrow (i+i) \\
            & \Rightarrow P+i\uparrow (i+i) \\
            & \Rightarrow F+i\uparrow (i+i) \\
            & \Rightarrow i+i\uparrow (i+i)
        \end{align*}

\section*{12-4}

    \subsection*{(1)}
        $E\uparrow$ 的有效项目集:
        $$\{\}$$

    \subsection*{(2)}
        $E-($ 的有效项目集:
        $$\{E.-(\}$$

    \subsection*{(3)}
        $E-T$ 的有效项目集:
        $$\{E-T.\}$$

\section*{12-5.1}
    \subsection*{1 构造 DFA}
        \subsubsection*{(1) 拓广文法}
            \begin{align*}
                r_1 = S &\to E \\
                r_2 = E &\to wX \\
                r_3 = E &\to xY \\
                r_4 = X &\to yX \\
                r_5 = X &\to z \\
                r_6 = Y &\to yY \\
                r_7 = Y &\to z
            \end{align*}

        \subsubsection*{(2) 列出项目}
            \begin{align*}
                S &\to .E \\
                S &\to E. \\
                E &\to .wX \\
                E &\to w.X \\
                E &\to wX. \\
                E &\to .xY \\
                E &\to x.Y \\
                E &\to xY. \\
                X &\to .yX \\
                X &\to y.X \\
                X &\to yX. \\
                X &\to .z \\
                X &\to z. \\
                Y &\to .yY \\
                Y &\to y.Y \\
                Y &\to yY. \\
                Y &\to .z \\
                Y &\to z.
            \end{align*}

        \subsubsection*{(3) 组成项目集}
            \begin{align*}
                I_0 = closure\{S\to .E\} &= \{S\to .E, E\to .wX, E\to .xY\} \\
                I_1 = closure\{S\to E.\} &= \{S\to E.\} \\
                I_2 = closure\{E\to w.X\} &= \{E\to w.X, X\to .yX, X\to .z\} \\
                I_3 = closure\{E\to x.Y\} &= \{E\to x.Y, Y\to .yY, Y\to .z\} \\
                I_4 = closure\{E\to wX.\} &= \{E\to wX.\} \\
                I_5 = closure\{X\to y.X\} &= \{X\to y.X, X\to .yX, X\to .z\} \\
                I_6 = closure\{X\to z.\} &= \{X\to z.\} \\
                I_7 = closure\{E\to xY.\} &= \{E\to xY.\} \\
                I_{8} = closure\{Y\to y.Y\} &= \{Y\to y.Y, Y\to .yY, Y\to .z\} \\
                I_{9} = closure\{Y\to z.\} &= \{Y\to z.\} \\
                I_{10} = closure\{X\to yX.\} &= \{X\to yX.\} \\
                I_{11} = closure\{Y\to yY.\} &= \{Y\to yY.\} \\
            \end{align*}
            上述所有均为 $LR(0)$ 的组成部分。

        \subsubsection*{(4) 构造 DFA}
            构造所得的 DFA 参图\ref{dfa}。
            \begin{figure}[htbp!]
                \centering
                \includegraphics[scale=0.1]{DFA.jpg}
                \caption{DFA图}
                \label{dfa}
            \end{figure}

    \subsection*{2 构造 LR(0) 转移表}

        GOTO 分析表参见表\ref{lr0goto}。
        \begin{table}[htbp!]
            \centering
            \begin{tabular}{|c|c|c|c|c|c|c|c|}
                \hline
                & \multicolumn{7}{|c|}{GOTO} \\
                \hline
                状态 & $w$ & $x$ & $y$ & $z$ & $E$ & $X$ & $Y$ \\
                \hline
                0    &  2  &  3  &     &     &  1  &     &     \\
                \hline
                1    &     &     &     &     &     &     &     \\
                \hline
                2    &     &     &  5  &  6  &     &  4  &     \\
                \hline
                3    &     &     &  8  &  9  &     &     & 11  \\
                \hline
                4    &     &     &     &     &     &     &     \\
                \hline
                5    &     &     &  5  &  6  &     & 10  &     \\
                \hline
                6    &     &     &     &     &     &     &     \\
                \hline
                7    &     &     &     &     &     &     &     \\
                \hline
                8    &     &     &  8  &  9  &     &     & 11  \\
                \hline
                9    &     &     &     &     &     &     &     \\
                \hline
                10   &     &     &     &     &     &     &     \\
                \hline
                11   &     &     &     &     &     &     &     \\
                \hline
            \end{tabular}
            \caption{该文法的 GOTO 分析表}
            \label{lr0goto}
        \end{table}    
    
        该文法的 LR(0) 分析表参见表\ref{lr0anal}。
        \begin{table}[htbp!]
            \centering
            \begin{tabular}{|c|c|c|c|c|c|c|c|c|}
                \hline
                & \multicolumn{7}{|c|}{GOTO} & Action\\
                \hline
                状态 & $w$ & $x$ & $y$ & $z$ & $E$ & $X$ & $Y$ &      \\
                \hline
                0    &$S_2$&$S_3$&     &     &$S_1$&     &     & Shift \\
                \hline
                1    &     &     &     &     &     &     &     & Accept \\
                \hline
                2    &     &     &$S_5$&$S_6$&     &$S_4$&     & Shift \\
                \hline
                3    &     &     &$S_8$&$S_9$&     &     &$S_{11}$& Shift \\
                \hline
                4    &     &     &     &     &     &     &     & $r_2$ \\
                \hline
                5    &     &     &$S_5$&$S_6$&     &$S_{10}$&     & Shift \\
                \hline
                6    &     &     &     &     &     &     &     & $r_5$ \\
                \hline
                7    &     &     &     &     &     &     &     & $r_3$ \\
                \hline
                8    &     &     &$S_8$&$S_9$&     &     &$S_{11}$& Shift \\
                \hline
                9    &     &     &     &     &     &     &     & $r_7$ \\
                \hline
                10   &     &     &     &     &     &     &     & $r_4$ \\
                \hline
                11   &     &     &     &     &     &     &     & $r_6$ \\
                \hline
            \end{tabular}
            \caption{该文法的 LR(0) 分析表}
            \label{lr0anal}
        \end{table}

\section*{12-5.2}
    \subsection*{(1)}
        $wyzz$ 的 LR(0) 分析过程参见表 \ref{wyyz}。
        \begin{table}[htbp!]
            \centering
            \begin{tabular}{|c|c|c|c|c|c|}
                \hline
                步骤 & 符号栈 & 输入串 & 活前缀 & 句柄 & 动作 \\
                \hline
                1 & $0$ & $wyyz$ &  &  & Shift \\
                \hline
                2 & $0w2$ & $yyz$ & $w$ & & Shift \\
                \hline
                3 & $0w2y5$ & $yz$ & $wy$ & & Shift \\
                \hline
                4 & $0w2y5y5$ & $z$ & $wyy$ & & Shift \\
                \hline
                5 & $0w2y5y5z6$ &  & $wyyz$ & $z$ & $X\to z$ \\
                \hline
                6 & $0w2y5y5X10$ &  & $wyyX$ & $yX$ & $X\to yX$ \\
                \hline
                7 & $0w2y5X10$ &  & $wyX$ & $yX$ & $X\to yX$ \\
                \hline
                8 & $0w2X4$ &  & $wX$ & $wX$ & $E\to wX$ \\
                \hline
                8 & $0E1$ &  &  &  & Accept \\
                \hline
            \end{tabular}
            \caption{$wyyz$的 LR(0) 分析过程}
            \label{wyyz}
        \end{table}
    
    \subsection*{(2)}
        $xyyyz$ 的 LR(0) 分析过程参见表 \ref{xyyyz}。
        \begin{table}[htbp!]
            \centering
            \begin{tabular}{|c|c|c|c|c|c|}
                \hline
                步骤 & 符号栈 & 输入串 & 活前缀 & 句柄 & 动作 \\
                \hline
                1 & $0$ & $xyyyz$ &  &  & Shift \\
                \hline
                2 & $0x3$ & $yyyz$ & $x$ &  & Shift \\
                \hline
                3 & $0x3y8$ & $yyz$ & $xy$ &  & Shift \\
                \hline
                4 & $0x3y8y8$ & $yz$ & $xyy$ &  & Shift \\
                \hline
                5 & $0x3y8y8y8$ & $z$ & $xyyy$ &  & Shift \\
                \hline
                6 & $0x3y8y8y8z9$ &  & $xyyyz$ & $z$ & $Y\to z$ \\
                \hline
                7 & $0x3y8y8Y11$ &  & $xyyY$ & $yY$ & $Y\to yY$ \\
                \hline
                8 & $0x3y8Y11$ &  & $xyY$ & $yY$ & $Y\to yY$ \\
                \hline
                9 & $0x3Y11$ &  & $xY$ & $xY$ & $E\to xY$ \\
                \hline
                10 & $0E1$ &  &  &  & Accept \\
                \hline
                
            \end{tabular}
            \caption{$xyyyz$的 LR(0) 分析过程}
            \label{xyyyz}
        \end{table}

\section*{12-5.6}
    根据和上题完全相同的做法,我们可以构造出该文法的 GOTO 表,参见表\ref{slr1goto}。
    草稿纸参见\ref{slr0}
    \begin{figure}[htbp!]
        \centering
        \includegraphics[scale=0.08]{LR0.jpg}
        \caption{SLR0 构造的草稿纸}
        \label{slr0}
    \end{figure}

    \begin{table}[htbp!]
        \centering
        \begin{tabular}{|c|c|c|c|c|c|c|c|c|c|c|}
            \hline
            & \multicolumn{10}{|c|}{GOTO}  \\
            \hline
            State &  $s$  &  $e$  &  $;$  &  $b$  &  $d$  &  $P$  &  $C$  &  $B$  &  $H$  &  $T$  \\
            \hline
            0     &       &       &       &   3   &       &   6   &  1    &   2   &   4   &       \\
            \hline
            1     &       &       &       &       &       &       &       &       &       &       \\
            \hline
            2     &       &       &       &       &       &       &       &       &       &       \\
            \hline
            3     &   7   &       &       &       &  13   &       &       &       &       &   5   \\
            \hline
            4     &       &       &   8   &       &       &       &       &       &       &       \\
            \hline
            5     &       &       &       &       &       &       &       &       &       &       \\
            \hline
            6     &       &       &       &       &       &       &       &       &       &       \\
            \hline
            7     &       &   9   &   10  &       &       &       &       &       &       &       \\
            \hline
            8     &   7   &       &       &       &  14   &       &       &       &       &       \\
            \hline
            9     &       &       &       &       &       &       &       &       &       &       \\
            \hline
            10    &   7   &       &       &       &       &       &       &       &       &   12  \\
            \hline
            11    &       &       &       &       &       &       &       &       &       &       \\
            \hline
            12    &       &       &       &       &       &       &       &       &       &       \\
            \hline
            13    &       &       &       &       &       &       &       &       &       &       \\
            \hline
            14    &       &       &       &       &       &       &       &       &       &       \\
            \hline
            
        \end{tabular}
        \caption{该文法的 GOTO 分析表}
        \label{slr1goto}
    \end{table}

    依序构造 FOLLOW 集如下:
    \begin{align*}
        FOLLOW(P) &= \{\#\} \\
        FOLLOW(C) &= \{\#\} \\
        FOLLOW(B) &= \{\#\} \\
        FOLLOW(T) &= \{\#\} \\
        FOLLOW(H) &= \{;\} \\
    \end{align*}

    因此可以推得其 SLR 分析表参表\ref{slr1}。

    \begin{table}[htbp!]
        \centering
        \begin{tabular}{|c|c|c|c|c|c|c|c|c|c|c|c|}
            \hline
            & \multicolumn{11}{|c|}{GOTO}  \\
            \hline
            State &  $s$  &  $e$  &  $;$  &  $b$  &  $d$  &  $\#$ &  $P$  &  $C$  &  $B$  &  $H$  &  $T$  \\
            \hline
            0     &       &       &       & $S_3$ &       &       &   6   &  1    &   2   &   4   &       \\
            \hline
            1     &       &       &       &       &       &  acc  &       &       &       &       &       \\
            \hline
            2     &       &       &       &       &       & $r_2$ &       &       &       &       &       \\
            \hline
            3     & $S_7$ &       &       &       &$S_{13}$&      &       &       &       &       & $S_5$ \\
            \hline
            4     &       &       & $S_8$ &       &       &       &       &       &       &       &       \\
            \hline
            5     &       &       & $r_5$ &       &       &       &       &       &       &       &       \\
            \hline
            6     &       &       &       &       &       & $r_3$ &       &       &       &       &       \\
            \hline
            7     &       & $S_9$ &$S_{10}$&      &       &       &       &       &       &       &       \\
            \hline
            8     & $S_7$ &       &       &       &$S_{14}$&      &       &       &       &       &       \\
            \hline
            9     &       &       &       &       &       & $r_7$ &       &       &       &       &       \\
            \hline
            10    & $S_7$ &       &       &       &       &       &       &       &       &       &  12   \\
            \hline
            11    &       &       &       &       &       & $r_4$ &       &       &       &       &       \\
            \hline
            12    &       &       &       &       &       & $r_8$ &       &       &       &       &       \\
            \hline
            13    &       &       &       &       &       & $r_5$ &       &       &       &       &       \\
            \hline
            14    &       &       &       &       &       & $r_6$ &       &       &       &       &       \\
            \hline
            
        \end{tabular}
        \caption{该文法的 SLR 分析表}
        \label{slr1}
    \end{table}

\section*{12-5.9}
    画出转移图后可以发现,$I_0$的第二项,使得$ACTION[0, a]$为移入2进展,因为$FOLLOW(Y)$包含$a$。$I_0$的第四项使得$ACTION[0, a]$为按$Y\to \varepsilon$规约,$ACTION[0, a]$有多重含义,故不是$SLR(1)$文法。

\section*{12-6}
    \subsection*{(1)}
        显然:
        \begin{align*}
            S &\to .E \\
            S &\to E. \\
            E &\to .E-T \\
            E &\to E.-T \\
            E &\to E-.T \\
            E &\to E-T. \\
            E &\to .T \\
            E &\to T. \\
            T &\to .T*F \\
            T &\to T.*F \\
            T &\to T*.F \\
            T &\to T*F. \\
            T &\to .F \\
            T &\to F. \\
            F &\to .i \\
            F &\to i. \\
            F &\to .(E) \\
            F &\to (.E) \\
            F &\to (E.) \\
            F &\to (E). \\
        \end{align*}
    \subsection*{(2)}
        分析表参表\ref{lr1table}
        \begin{table}[htbp!]
            \centering
            \begin{tabular}{|c|c|c|c|c|c|c|c|c|c|}
                \hline
                & \multicolumn{6}{|c|}{Action} & \multicolumn{3}{|c|}{GOTO} \\
                \hline
                State & $-$ & $*$ & $i$ & $($ & $)$ & $\#$ & $E$ & $T$ & $F$ \\
                \hline
                0 & & $S_4$ & $S_5$ & & & & 1 & 2 & 3 \\
                \hline
                1 & $S_6$ & & & & & acc & & & \\
                \hline
                2 & $r_3$ & $S_7$ & & & & $r_3$ & & & \\
                \hline
                3 & $r_5$ & $r_5$ & & & & $r_5$ & & & \\
                \hline
                4 & $r_6$ & $r_6$ & & & & $r_6$ & & & \\
                \hline
                5 & & & $S_{11}$ & $S_{12}$ & & & 8 & 9 & 10 \\
                \hline
                6 & & & $S_4$ & $S_5$ & & & & 13 & 3 \\
                \hline
                7 & & & $S_4$ & $S_5$ & & & & & 14 \\
                \hline
                8 & $S_{16}$ & & & & $S_{15}$ & & & & \\
                \hline
                9 & $r_3$ & $S_{17}$ & & & $r_3$ & & & & \\
                \hline
                10 & $r_5$ & $r_5$ & & & $r_5$ & & & & \\
                \hline
                11 & $r_6$ & $r_6$ & & & $r_6$ & & & & \\
                \hline
                12 & & & $S_{11}$ & $S_{12}$ & & & & & \\
                \hline
                13 & $r_2$ & $r_7$ & & & $r_2$ & & & & \\
                \hline
                14 & $r_4$ & $r_4$ & & & $r_4$ & & & & \\
                \hline
                15 & $r_7$ & $r_7$ & & & $r_7$ & & & & \\
                \hline
                16 & & & $S_{11}$ & $S_{12}$ & & & & 19 & 10 \\
                \hline
                17 & & & $S_{11}$ & $S_{12}$ & & & & & 20 \\
                \hline
                18 & $S_{16}$ & & & & $S_{21}$ & & & & \\
                \hline
                19 & $r_2$ & $S_17$ & & & $r_2$ & & & & \\
                \hline
                20 & $r_4$ & $r_4$ & & & $r_4$ & & & & \\
                \hline
                21 & $r_7$ & $r_7$ & & & $r_7$ & & & & \\
                \hline
            \end{tabular}
            \caption{该文法的 SLR 分析表}
            \label{lr1table}
        \end{table}
    \subsection*{(3)}

        $(i-i)* i$ 的 LR(1) 分析过程参见表\ref{pos}。
        \begin{table}[htbp!]
            \centering
            \begin{tabular}{|c|c|c|c|c|c|}
                \hline
                步骤 & 符号栈 & 输入串 & 活前缀 & 句柄 \\
                \hline
                \hline
                1 & $0$ & $(i-i)*i\#$ &  &   \\
                \hline
                2 & $0(5$ & $i-i)*i\#$ & $($  &   \\
                \hline
                3 & $0(5i11$ & $-i)*i\#$ & $(i$  & $i$ \\
                \hline
                4 & $0(5F10$ & $-i)*i\#$ & $(F$  & $F$ \\
                \hline
                5 & $0(5F9$ & $-i)*i\#$ & $(T$  & $T$ \\
                \hline
                6 & $0(5E8$ & $-i)*i\#$ & $(E$  & \\
                \hline
                7 & $0(5E8-16$ & $i)*i\#$ & $(E-$  & \\
                \hline
                8 & $0(5E8-16i11$ & $)*i\#$ & $(E-i$  & $i$ \\
                \hline
                9 & $0(5E8-16F10$ & $)*i\#$ & $(E-F$  & $F$\\
                \hline
                10 & $0(5E8-16T19$ & $)*i\#$ & $(E-T$  & $E-T$ \\
                \hline
                11 & $0(5E8$ & $)*i\#$ & $(E$  &  \\
                \hline
                12 & $0(5E8)15$ & $*i\#$ & $(E)$  & $(E)$ \\
                \hline
                13 & $0F3$ & $*i\#$ & $F$  & $F$ \\
                \hline
                14 & $0T2$ & $*i\#$ & $T$  & \\
                \hline
                15 & $0T2*7$ & $i\#$ & $T*$  & \\
                \hline
                16 & $0T2*7i4$ & $\#$ & $T*i$  & $i$ \\
                \hline
                17 & $0T2*7F14$ & $\#$ & $T*F$  & $T*F$ \\
                \hline
                18 & $0T2$ & $\#$ & $T$  & $T$ \\
                \hline
                19 & $0E1$ & $\#$ &  & \\
                \hline
                
            \end{tabular}
            \caption{$(i-i)* i$的 LR(1) 分析过程}
            \label{pos}
        \end{table}

        $i-i*i$ 的分析过程参见表\ref{sop}
        \begin{table}[htbp!]
            \centering
            \begin{tabular}{|c|c|c|c|c|c|}
                \hline
                步骤 & 符号栈 & 输入串 & 活前缀 & 句柄 \\
                \hline
                1 & $0$ & $i-i*i\#$ &  &   \\
                \hline
                2 & $0i4$ & $-i*i\#$ & $i$ & $i$  \\
                \hline
                3 & $0F3$ & $-i*i\#$ & $F$ & $F$  \\
                \hline
                4 & $0T2$ & $-i*i\#$ & $T$ & $T$  \\
                \hline
                5 & $0E1$ & $-i*i\#$ & $E$ &  \\
                \hline
                6 & $0E1-6$ & $i*i\#$ & $E-$ &  \\
                \hline
                7 & $0E1-6i4$ & $*i\#$ & $E-i$ & $i$ \\
                \hline
                8 & $0E1-6F3$ & $*i\#$ & $E-F$ & $F$ \\
                \hline
                9 & $0E1-6T13$ & $*i\#$ & $E-T$ & \\
                \hline
                10 & $0E1-6T13*7$ & $i\#$ & $E-T*$ & \\
                \hline
                11 & $0E1-6T13*7i4$ & $\#$ & $E-T*i$ & $i$ \\
                \hline
                12 & $0E1-6T13*7F14$ & $\#$ & $E-T*F$ & $T*F$ \\
                \hline
                13 & $0E1-6T13$ & $\#$ & $E-T$ & $E-T$ \\
                \hline
                14 & $0E1$ & $\#$ & & \\
                \hline
            \end{tabular}
            \caption{$i-i*i$的 LR(1) 分析过程}
            \label{sop}
        \end{table}

\end{document}