\documentclass[UTF8]{ctexart}
\usepackage{amsmath,amssymb}
\usepackage{fancyhdr}
\usepackage{amsmath,bm}
\usepackage{mathrsfs}
\usepackage{ntheorem}
\usepackage{graphicx}
\usepackage{subfigure}
\usepackage[top=2cm, bottom=2cm, left=2cm, right=2cm]{geometry}  
\usepackage{algorithm}  
\usepackage{algorithmicx}  
\usepackage{algpseudocode}
\usepackage{multirow}
\usepackage{tikz}
\usepackage{listings}
\usepackage{xcolor}
\usetikzlibrary{automata, positioning, arrows}
\tikzset{
    ->,
    >=stealth,
    node distance = 3cm,
    every state/.style={thick, fill=gray!10},
    initial text=$ $
}
\lstset{numbers=left, %设置行号位置
        numberstyle=\tiny, %设置行号大小
        keywordstyle=\color{blue}, %设置关键字颜色
        commentstyle=\color[cmyk]{1,0,1,0}, %设置注释颜色
        frame=single, %设置边框格式
        escapeinside=``, %逃逸字符(1左面的键),用于显示中文
        %breaklines, %自动折行
        extendedchars=false, %解决代码跨页时,章节标题,页眉等汉字不显示的问题
        xleftmargin=2em,xrightmargin=2em, aboveskip=1em, %设置边距
        tabsize=4, %设置tab空格数
        showspaces=false %不显示空格
    }
\floatname{algorithm}{算法}  
\renewcommand{\algorithmicrequire}{\textbf{输入:}}  
\renewcommand{\algorithmicensure}{\textbf{输出:}}  

\theorembodyfont{\normalfont\rm\CJKfamily{song}}
%\theoremstyle{break}
\newtheorem{theorem}{定理}
\newtheorem{lemma}{引理}
\newtheorem{proposition}{命题}
\newtheorem*{proof}{证}[section]
\newtheorem*{solution}{解}[section]
\title{语法分析作业}
\author{丁元杰 17231164}
\date{\today}

\begin{document}
\maketitle

\section*{4-2.1}
\subsection*{(1)}
要求:
\begin{itemize}
    \item 文法中不存在左递归。左递归的文法无法找到递归边界。
    \item 每一个非终结符的FIRST集合两两不相交。否则算法需要回溯。
\end{itemize}

\subsection*{(2)}
自顶向下的分析方法需要依据每个分支的第一个符号的FIRST集合来判断往何处递归调用,左递归的文法中,左递归的部分无法找出这样的FIRST集合;而其他的文法即使有递归,只要递归的部分不在分支的第一个位置,就可以找到其FIRST集合,从而决定递归的方向,因而可以被自顶向下的分析方法分析语法。

\section*{4-2.2}
\lstinputlisting {4-2-2.py}

\section*{4-2.3}
\subsection*{(1)}
\begin{align*}
    FIRST(AcB) & = \{c\} \\
    FIRST(Bd) & = \{a\} \\
    FIRST(AaB) &= \{c\} \\
    FIRST(c) &= \{c\} \\
    FIRST(aA) &= \{a\} \\
    FIRST(a) &= \{a\}
\end{align*}

\subsection*{(2)}
该文法并不能用自顶向下的分析方法进行分析。因为
\begin{itemize}
    \item 存在左递归
    \item FIRST集合存在交
\end{itemize}

\subsection*{(3)}
新文法如下:
\begin{align*}
    Z&::=AcB \big| Bd\\
    A&::=c\{aB\}\\
    B&::=a[A]
\end{align*}

\lstinputlisting {4-2-3.py}

\end{document}