\documentclass[UTF8]{ctexart}
\usepackage{amsmath,amssymb}
\usepackage{fancyhdr}
\usepackage{amsmath,bm}
\usepackage{mathrsfs}
\usepackage{ntheorem}
\usepackage{graphicx}
\usepackage{subfigure}
\usepackage[top=2cm, bottom=2cm, left=2cm, right=2cm]{geometry}  
\usepackage{algorithm}  
\usepackage{algorithmicx}  
\usepackage{algpseudocode}
\usepackage{multirow}
\usepackage{tikz}
\usepackage{listings}
\usepackage{palatino}
\usepackage{xcolor}
\usepackage{syntax}
\usetikzlibrary{automata, positioning, arrows, shapes.geometric}
\tikzstyle{startstop} = [rectangle, minimum width = 2cm, minimum height=1cm,text centered, draw = black]
\tikzstyle{io} = [trapezium, trapezium left angle=70, trapezium right angle=110, minimum width=2cm, minimum height=1cm, text centered, draw=black]
\tikzstyle{process} = [rectangle, minimum width=3cm, minimum height=1cm, text centered, draw=black]
\tikzstyle{arrow} = [->,>=stealth]
\tikzstyle{decision} = [diamond, aspect = 3, text centered, draw=black]
\tikzset{
    ->,
    >=stealth,
    node distance = 3cm,
    every state/.style={thick, fill=gray!10},
    initial text=$ $
}
\lstset{numbers=left, %设置行号位置
        numberstyle=\tiny, %设置行号大小
        keywordstyle=\color{blue}, %设置关键字颜色
        commentstyle=\color[cmyk]{1,0,1,0}, %设置注释颜色
        frame=single, %设置边框格式
        escapeinside=``, %逃逸字符(1左面的键),用于显示中文
        %breaklines, %自动折行
        extendedchars=false, %解决代码跨页时,章节标题,页眉等汉字不显示的问题
        xleftmargin=2em,xrightmargin=2em, aboveskip=1em, %设置边距
        tabsize=4, %设置tab空格数
        showspaces=false %不显示空格
    }
\setlength{\grammarparsep}{10pt plus 1pt minus 1pt} % increase separation between rules
\setlength{\grammarindent}{20em} % increase separation between LHS/RHS 
\floatname{algorithm}{算法}  
\renewcommand{\algorithmicrequire}{\textbf{输入:}}  
\renewcommand{\algorithmicensure}{\textbf{输出:}}  

\theorembodyfont{\normalfont\rm\CJKfamily{song}}
%\theoremstyle{break}
\newtheorem{theorem}{定理}
\newtheorem{lemma}{引理}
\newtheorem{proposition}{命题}
\newtheorem*{proof}{证}[section]
\newtheorem*{solution}{解}[section]
\title{高级语法分析}
\author{丁元杰 17231164}
\date{\today}

\begin{document}
\maketitle

% 箭头形式

\section*{12.1}

    \subsection*{(1)}
        构造如下:
        \begin{align*}
            FIRST(E) &= \{(, a, b, \wedge\} \\
            FIRST(T) &= \{(, a, b, \wedge\} \\
            FIRST(F) &= \{(, a, b, \wedge\} \\
            FIRST(P) &= \{(, a, b, \wedge\} \\
            FIRST(T') &= \{(, a, b, \wedge, \varepsilon\} \\
            FIRST(E') &= \{+, \varepsilon\} \\
            FIRST(F') &= \{*\} \\
            FOLLOW(E) &= \{\#, )\} \\
            FOLLOW(E') &= \{\#, )\} \\
            FOLLOW(T) &= \{+\} \\
            FOLLOW(T') &= \{+\} \\
            FOLLOW(F) &= \{(, a, b, \wedge\} \\
            FOLLOW(F') &= \{(, a, b, \wedge\} \\
            FOLLOW(P) &= \{(, a, b, \wedge\} \\
        \end{align*}

    \subsection*{(2)}
        显然的有:
        \begin{align*}
            \{+\} \cap \{\varepsilon\} &= \emptyset \\
            \{(, a, b, \wedge\} \cap \{\varepsilon\} &= \emptyset \\
            \{*\} \cap \{\varepsilon\} &= \emptyset
        \end{align*}

        所以文法$G[E]$是LL(1)的。

    \subsection*{(3)}

        预测分析表见下表:

        \begin{tabular}{|c|c|c|c|c|c|c|c|c|}
            \hline
            & + & ( & ) & $*$ & $\wedge$ & $a$ & $b$ & \# \\
            \hline
            $E$ & & $E\to TE'$ & & & $E\to TE'$ & $E\to TE'$ & $E\to TE'$ & \\
            \hline
            $E'$ & $E\to +$ & & $E\to \varepsilon$ & & & & & $E\to \varepsilon$ \\
            \hline
            $T$ & & $T\to FT'$ & & & $T\to FT'$ & $T\to FT'$ & $T\to FT'$ & \\
            \hline
            $T'$ & $T'\to \varepsilon$ & $T'\to T$ & & & $T'\to T$ & $T'\to T$ & $T'\to T$ & \\
            \hline
            $F$ & & $F\to PF'$ & & & $F\to PF'$ &$F\to PF'$ &$F\to PF'$ & \\
            \hline
            $F'$ & & $F'\to \varepsilon$ & & $F'\to *F'$ & $F'\to \varepsilon$ &$F'\to \varepsilon$ &$F'\to \varepsilon$ & \\
            \hline
            $P$ & & $P\to (E)$ & & & $P\to\wedge$ & $P\to a$ & $P\to b$ & \\
            \hline
        \end{tabular}

\section*{12.2}

    \subsection*{(1)}
        FOLLOW 集合如下:
        \begin{align*}
            FOLLOW(S) &= \{d, a, f, \#\} \\
            FOLLOW(A) &= \{a, d, e\} \\
            FOLLOW(B) &= \{b\} \\
            FOLLOW(C) &= \{b, g\}
        \end{align*}

    \subsection*{(2)}
        FIRST 集合如下:
        \begin{align*}
            FIRST(aABbcd) &= \{a\} \\
            FIRST(ASd) &= \{a\} \\
            FIRST(Sah) &= \{a\} \\
            FIRST(eC) &= \{e\} \\
            FIRST(Sf) &= \{a\} \\
            FIRST(Cg) &= \{a\} \\
        \end{align*}

    \subsection*{(3)}
        此文法并非 LL(1) 的文法,原因如下:
        \begin{enumerate}
            \item 产生式中存在左递归,
            \item $C$的产生式中,$Sf$和$Cg$的FIRST集合相交了。
        \end{enumerate}

\section*{12.6}
    不会做

\section*{补充作业}
    \subsection*{(1)}
        对于句子$ibtibtaea$,有符合此文法的两种语法树:
        \begin{enumerate}
            \item $[i, [b], [i, [b], t, [a], [e, [a]]]]$
            \item $[i, [b], [i, [b], t, [a]], [e, [a]]]$
        \end{enumerate}
        所以此文法是二义的。

    \subsection*{(2)}
        \begin{align*}
            FIRST(S) &= \{i, a\} \\
            FIRST(S') &= \{e, \varepsilon\} \\
            FIRST(C) &= \{b\} \\
            FOLLOW(S) &= \{\#, e\} \\
            FOLLOW(S') &= \{\#, e\} \\
            FOLLOW(C) &= \{t\} 
        \end{align*}
    
    \subsection*{(3)}
        构造的表格如下:
        
        \begin{tabular}{|c|c|c|c|c|c|c|}
            \hline
            & $i$ & $t$ & $e$ & $a$ & $b$ & $\#$ \\
            \hline
            $S$ & $S\to iCtSS'$ & & & $S\to a$ & & \\
            \hline
            $S'$ & & & $S'\to \varepsilon$ & & & $S'\to \varepsilon$ \\
            \hline
            $C$ & & & & $C\to b$ & \\
            \hline
        \end{tabular}

\end{document}