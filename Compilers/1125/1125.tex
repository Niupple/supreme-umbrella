\documentclass[UTF8]{ctexart}
\usepackage{amsmath,amssymb}
\usepackage{fancyhdr}
\usepackage{amsmath,bm}
\usepackage{mathrsfs}
\usepackage{ntheorem}
\usepackage{graphicx}
\usepackage{subfigure}
\usepackage[top=2cm, bottom=2cm, left=2cm, right=2cm]{geometry}  
\usepackage{algorithm}  
\usepackage{algorithmicx}  
\usepackage{algpseudocode}
\usepackage{multirow}
\usepackage{tikz}
\usepackage{listings}
\usepackage{palatino}
\usepackage{xcolor}
\usepackage{syntax}
\usetikzlibrary{automata, positioning, arrows, shapes.geometric}
\tikzstyle{startstop} = [rectangle, minimum width = 2cm, minimum height=1cm,text centered, draw = black]
\tikzstyle{io} = [trapezium, trapezium left angle=70, trapezium right angle=110, minimum width=2cm, minimum height=1cm, text centered, draw=black]
\tikzstyle{process} = [rectangle, minimum width=3cm, minimum height=1cm, text centered, draw=black]
\tikzstyle{arrow} = [->,>=stealth]
\tikzstyle{decision} = [diamond, aspect = 3, text centered, draw=black]
\tikzset{
    ->,
    >=stealth,
    node distance = 3cm,
    every state/.style={thick, fill=gray!10},
    initial text=$ $
}
\lstset{numbers=left, %设置行号位置
        numberstyle=\tiny, %设置行号大小
        keywordstyle=\color{blue}, %设置关键字颜色
        commentstyle=\color[cmyk]{1,0,1,0}, %设置注释颜色
        frame=single, %设置边框格式
        escapeinside=``, %逃逸字符(1左面的键),用于显示中文
        %breaklines, %自动折行
        extendedchars=false, %解决代码跨页时,章节标题,页眉等汉字不显示的问题
        xleftmargin=2em,xrightmargin=2em, aboveskip=1em, %设置边距
        tabsize=4, %设置tab空格数
        showspaces=false %不显示空格
    }
\setlength{\grammarparsep}{10pt plus 1pt minus 1pt} % increase separation between rules
\setlength{\grammarindent}{20em} % increase separation between LHS/RHS 
\floatname{algorithm}{算法}  
\renewcommand{\algorithmicrequire}{\textbf{输入:}}  
\renewcommand{\algorithmicensure}{\textbf{输出:}}  

\theorembodyfont{\normalfont\rm\CJKfamily{song}}
%\theoremstyle{break}
\newtheorem{theorem}{定理}
\newtheorem{lemma}{引理}
\newtheorem{proposition}{命题}
\newtheorem*{proof}{证}[section]
\newtheorem*{solution}{解}[section]
\title{高级语法分析}
\author{丁元杰 17231164}
\date{\today}

\begin{document}
\maketitle

% 箭头形式

\section*{12-2.2}

    \subsection*{(2)}
        \begin{tabular}{|c|c|c|c|c|}
            \hline
            步骤 & 符号栈 & 优先关系 & 读入符号 & 输入串 \\
            \hline
            $0$ & $\#$ & $<$ & $a$ & $+b*(c+d)-e\#$ \\
            \hline
            $1$ & $\#a$ & $>$ & $+$ & $b*(c+d)-e\#$ \\
            \hline
            $2$ & $\#E$ & $<$ & $+$ & $b*(c+d)-e\#$ \\
            \hline
            $3$ & $\#E+$ & $<$ & $b$ & $*(c+d)-e\#$ \\
            \hline
            $4$ & $\#E+b$ & $>$ & $*$ & $(c+d)-e\#$ \\
            \hline
            $4$ & $\#E+E$ & $<$ & $*$ & $(c+d)-e\#$ \\
            \hline
            $5$ & $\#E+E*$ & $<$ & $($ & $c+d)-e\#$ \\
            \hline
            $6$ & $\#E+E*($ & $<$ & $c$ & $+d)-e\#$ \\
            \hline
            $7$ & $\#E+E*(c$ & $>$ & $+$ & $d)-e\#$ \\
            \hline
            $7$ & $\#E+E*(E$ & $<$ & $+$ & $d)-e\#$ \\
            \hline
            $8$ & $\#E+E*(E+$ & $<$ & $d$ & $)-e\#$ \\
            \hline
            $9$ & $\#E+E*(E+d$ & $>$ & $)$ & $-e\#$ \\
            \hline
            $9$ & $\#E+E*(E+E$ & $>$ & $)$ & $-e\#$ \\
            \hline
            $9$ & $\#E+E*(E$ & $=$ & $)$ & $-e\#$ \\
            \hline
            $9$ & $\#E+E*(E)$ & $>$ & $-$ & $e\#$ \\
            \hline
            $9$ & $\#E+E*E$ & $>$ & $-$ & $e\#$ \\
            \hline
            $9$ & $\#E+E$ & $>$ & $-$ & $e\#$ \\
            \hline
            $9$ & $\#E$ & $<$ & $-$ & $e\#$ \\
            \hline
            $9$ & $\#E-$ & $<$ & $e$ & $\#$ \\
            \hline
            $9$ & $\#E-e$ & $>$ & $\#$ & \\
            \hline
            $9$ & $\#E-E$ & $>$ & $\#$ & \\
            \hline
            $9$ & $\#E$ & $>$ & $\#$ & \\
            \hline
            $9$ & $\#E$ & 停止 & $\#$ & \\
            \hline
        \end{tabular}

\section*{12-2.4}

    $E$:
    \begin{itemize}
        \item 短语:$E$
        \item 素短语:无
    \end{itemize}

    $T$:
    \begin{itemize}
        \item 短语:$T$
        \item 素短语:无
    \end{itemize}
    
    $i$:
    \begin{itemize}
        \item 短语:$i$
        \item 素短语:$i$
    \end{itemize}
    
    $T*F$:
    \begin{itemize}
        \item 短语:$T*F$
        \item 素短语:$T*F$
    \end{itemize}
    
    $F*F$:
    \begin{itemize}
        \item 短语:$F, F*F$
        \item 素短语:$F * F$
    \end{itemize}
    
    $i*F$:
    \begin{itemize}
        \item 短语:$i, i*F$
        \item 素短语:$i, i*F$
    \end{itemize}
    
    $F*i$:
    \begin{itemize}
        \item 短语:$F*i, F, i$
        \item 素短语:$F*i, i$
    \end{itemize}
    
    $F + F + F$:
    \begin{itemize}
        \item 短语:$F + F + F, F + F, F$
        \item 素短语:$F + F$
    \end{itemize}

\section*{12-2.5}

$i$

\begin{tabular}{|c|c|c|c|c|}
    \hline
    步骤 & 句型 & 优先关系 & 最左子串 & 规约符号 \\
    \hline
    1 & $\#i\#$ & $\# < i > \#$ & $i$ & $E$ \\
    \hline
\end{tabular}

$i+i$

\begin{tabular}{|c|c|c|c|c|}
    \hline
    步骤 & 句型 & 优先关系 & 最左子串 & 规约符号 \\
    \hline
    1 & $\#i+i\#$ & $\# < i > + < i > \#$ & $i$ & $F$ \\
    \hline
    1 & $\#F+i\#$ & $\# < + < i > \#$ & $i$ & $F$ \\
    \hline
    1 & $\#F+F\#$ & $\# <  +  > \#$ & $F+F$ & $E$ \\
    \hline
\end{tabular}

$i*i+i$

\begin{tabular}{|c|c|c|c|c|}
    \hline
    步骤 & 句型 & 优先关系 & 最左子串 & 规约符号 \\
    \hline
    1 & $\#i*i+i\#$ & $\# < i > * < i > + < i > \#$ & $i * i$ & $T$ \\
    \hline
    1 & $\#F*i+i\#$ & $\# < * < i > + < i > \#$ & $i$ & $F$ \\
    \hline
    1 & $\#F*F+i\#$ & $\# < * > + < i > \#$ & $F*F$ & $T$ \\
    \hline
    1 & $\#T+i\#$ & $\# < + < i > \#$ & $i$ & $F$ \\
    \hline
    1 & $\#T+F\#$ & $\# < + > \#$ & $T+F$ & $E$ \\
    \hline
\end{tabular}

$i*(i+i*i)+((i+i)*i)$
参见表\ref{induction}

\begin{table}
    \caption{规约表格}
    \centering
    \begin{tabular}{|p{30pt}|p{100pt}|p{200pt}|p{50pt}|p{30pt}|}
        \hline
        步骤 & 句型 & 优先关系 & 最左子串 & 规约符号 \\
        \hline
        1 & $\#i*(i+i*i)+((i+i)*i)\#$ & $\# < i > * < ( < i > + < i > * < i > ) > + < ( < ( < i > + < i > ) > * < i > ) > \#$ & $i$ & $F$ \\
        \hline
        1 & $\#F*(i+i*i)+((i+i)*i)\#$ & $\# < * < ( < i > + < i > * < i > ) > + < ( < ( < i > + < i > ) > * < i > ) > \#$ & $i$ & $F$ \\
        \hline
        1 & $\#F*(F+i*i)+((i+i)*i)\#$ & $\# < * < ( < + < i > * < i > ) > + < ( < ( < i > + < i > ) > * < i > ) > \#$ & $i$ & $F$ \\
        \hline
        1 & $\#F*(F+F*i)+((i+i)*i)\#$ & $\# < * < ( < + < * < i > ) > + < ( < ( < i > + < i > ) > * < i > ) > \#$ & $i$ & $F$ \\
        \hline
        1 & $\#F*(F+F*F)+((i+i)*i)\#$ & $\# < * < ( < + < * > ) > + < ( < ( < i > + < i > ) > * < i > ) > \#$ & $F*F$ & $T$ \\
        \hline
        1 & $\#F*(F+T)+((i+i)*i)\#$ & $\# < * < ( < + > ) > + < ( < ( < i > + < i > ) > * < i > ) > \#$ & $F+T$ & $E$ \\
        \hline
        1 & $\#F*(E)+((i+i)*i)\#$ & $\# < * < ( = ) > + < ( < ( < i > + < i > ) > * < i > ) > \#$ & $(E)$ & $F$ \\
        \hline
        1 & $\#F*F+((i+i)*i)\#$ & $\# < * > + < ( < ( < i > + < i > ) > * < i > ) > \#$ & $F*F$ & $T$ \\
        \hline
        1 & $\#T+((i+i)*i)\#$ & $\# < + < ( < ( < i > + < i > ) > * < i > ) > \#$ & $i$ & $F$ \\
        \hline
        1 & $\#T+((F+i)*i)\#$ & $\# < + < ( < ( < + < i > ) > * < i > ) > \#$ & $i$ & $F$ \\
        \hline
        1 & $\#T+((F+F)*i)\#$ & $\# < + < ( < ( < + > ) > * < i > ) > \#$ & $F+F$ & $E$ \\
        \hline
        1 & $\#T+((E)*i)\#$ & $\# < + < ( < ( = ) > * < i > ) > \#$ & $(E)$ & $F$ \\
        \hline
        1 & $\#T+(F*i)\#$ & $\# < + < ( < * < i > ) > \#$ & $i$ & $F$ \\
        \hline
        1 & $\#T+(F*F)\#$ & $\# < + < ( < * > ) > \#$ & $F*F$ & $T$ \\
        \hline
        1 & $\#T+(T)\#$ & $\# < + < ( = ) > \#$ & $(T)$ & $F$ \\
        \hline
        1 & $\#T+F\#$ & $\# < + > \#$ & $T+F$ & $E$ \\
        \hline
    \end{tabular}
    \label{induction}
\end{table}


\section*{补充作业}
    \subsection*{(1)}
        构造如下:
        \begin{align*}
            FIRSTVT(E) &= \{+\} \\
            FIRSTVT(T) &= \{(, i\} \\
            LASTVT(E) &= \{+\} \\
            LASTVT(T) &= \{), i\} 
        \end{align*}

    \subsection*{(2)}
        参见表\ref{matrix}.
        \begin{table}
            \caption{优先关系矩阵}
            \centering
            \begin{tabular}{|c||c|c|c|c|c|}
                \hline
                & $+$ & $i$ & $($ & $)$ & $\#$ \\
                \hline \hline
                $+$ & $>$ & $<$ & $<$ & $>$ & $>$ \\
                \hline
                $i$ & $>$ &  &  & $>$ & $>$ \\
                \hline
                $($ & $<$ & $<$ & $<$ & $=$ &  \\
                \hline
                $)$ & $>$ & $>$ &  & $>$ & $>$ \\
                \hline
                $\#$ & $<$ & $<$ & $<$ &  &  \\
                \hline
            \end{tabular}
            \label{matrix}
        \end{table}

    \subsection*{(3)}
        显然地,该文法是算符优先算法。

\end{document}